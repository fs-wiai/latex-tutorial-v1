\newcommand{\lang}{ngerman}

\ifdefined\ishandout
\newcommand{\handoutmode}{handout}
\else
\newcommand{\handoutmode}{}
\fi


\documentclass[10pt,
\lang ,
\handoutmode ,
compress
]{beamer}
% Für den Header
% Modify for different languages
\usepackage{ifthen}

\newcommand{\unibastring}{\ifthenelse{\equal{\lang}{ngerman}}{Universit\"at Bamberg}{University of Bamberg}}


\usepackage{eurosym}
\usepackage{etex}
\usepackage{ulem}
\usepackage{stmaryrd}
\usetheme{UniBa}
%\usefonttheme{
%	default | professionalfonts | serif |
%	structurebold | structureitalicserif |
%	structuresmallcapsserif
%}
\usefonttheme{professionalfonts}
%\useinnertheme{
%	circles | default | inmargin |
%	rectangles | rounded
%}
\useinnertheme{rectangles}
%\useoutertheme{
%	default | infolines | miniframes |
%	shadow | sidebar | smoothbars |
%	smoothtree | split | tree
%}
%\useoutertheme{split}
\setbeamercovered{transparent}

% Without navigation symbols
\beamertemplatenavigationsymbolsempty

%% Formatierungen
\usepackage{url}
\usepackage{latexsym}			% schönere Symbole
\usepackage{color}
%\usepackage{float}

%% Zeichensätze
\usepackage[utf8]{inputenc}
\usepackage{lmodern}
\usepackage{float}
%\usepackage{thumbpdf}
\usepackage{wasysym}
%\usepackage{ucs}

%Meta info
%Necessary Information
\author[C. Zeck, J. Karp, L. Dietz, M. Trager, V. Barth]{Christian Zeck, Jascha Karp, Linus Dietz, Michael Träger, Valentin Barth}
\title{\LaTeX ~-Tutorium}
%The day of the presentation
\date{\today}

%Optional Information
\subject{subject}
\keywords{keywords}

%Already set
\ifthenelse{\equal{\lang}{ngerman}}{%
\institute[FS WIAI]{Fachschaft Wirtschaftsinformatik und Angewandte Informatik\\ der Otto Friedrich Universit\"at Bamberg}}{%
\institute[KTR]{Professorship for Computer Science,\\%
        Communication Services, Telecommunication Systems and Computer Networks}}

\titlegraphic{\includegraphics[width=13mm,height=13mm]{image/logo}}


%% Hyperref
\usepackage{hyperref}

\makeatletter
\hypersetup{pdftitle={\@title}, pdfauthor={\@author}, linktoc=page, pdfborder={0 0 0 [3 3]}, breaklinks=true, linkbordercolor=unibablueI, menubordercolor=unibablueI, urlbordercolor=unibablueI, citebordercolor=unibablueI, filebordercolor=unibablueI}
\makeatother
%% Define a new 'leo' style for the package that will use a smaller font.
\makeatletter
\def\url@leostyle{%
  \@ifundefined{selectfont}{\def\UrlFont{\sf}}{\def\UrlFont{\small\ttfamily}}}
\makeatother
%% Now actually use the newly defined style.
\urlstyle{leo}

%% Sprache
\ifthenelse{\equal{\lang}{ngerman}}{\usepackage[german,ngerman]{babel}}{\usepackage[\lang]{babel}}
%\mode<presentation>{
%% XXX without this the number does not appear
%\AtBeginDocument{\def\figurename{{\scshape Fig.~\thefigure}}}
%}
%%\usepackage{abstract}

%% Mathe und Formeln
\usepackage{calc}
\usepackage{amsmath}
\usepackage{amssymb,amsthm,amsfonts}
\usepackage{dsfont}
\usepackage[nice]{nicefrac}
\usepackage{cancel}  %%druchstreichen von Formeln
%
%% Programmieren mit Latex
\usepackage{ifthen}


\usepackage{dirtree}   %setzen von baumstrukturen

%%%   Fuer anspruchsvolle Tabellen   %%
\usepackage{longtable, colortbl}
\usepackage{multicol, multirow}
%
%%%  Für Grafiken %%
\usepackage{graphicx}
\usepackage{tikz}
%\usepackage{pgfplots}
\usetikzlibrary{calc,arrows,fit,positioning,trees,backgrounds,shadows,decorations,decorations.markings,decorations.shapes,shapes,patterns,fadings}
\usepackage[font=footnotesize]{subfig}


\makeatletter
\newcount\dirtree@lvl
\newcount\dirtree@plvl
\newcount\dirtree@clvl
\def\dirtree@growth{%
  \ifnum\tikznumberofcurrentchild=1\relax
  \global\advance\dirtree@plvl by 1
  \expandafter\xdef\csname dirtree@p@\the\dirtree@plvl\endcsname{\the\dirtree@lvl}
  \fi
  \global\advance\dirtree@lvl by 1\relax
  \dirtree@clvl=\dirtree@lvl
  \advance\dirtree@clvl by -\csname dirtree@p@\the\dirtree@plvl\endcsname
  \pgf@xa=1cm\relax
  \pgf@ya=-1cm\relax
  \pgf@ya=\dirtree@clvl\pgf@ya
  \pgftransformshift{\pgfqpoint{\the\pgf@xa}{\the\pgf@ya}}%
  \ifnum\tikznumberofcurrentchild=\tikznumberofchildren
  \global\advance\dirtree@plvl by -1
  \fi
}

\tikzset{
  dirtree/.style={
    growth function=\dirtree@growth,
    every node/.style={anchor=north},
    every child node/.style={anchor=west},
    edge from parent path={(\tikzparentnode\tikzparentanchor) |- (\tikzchildnode\tikzchildanchor)}
  }
}
\makeatother
%\usepackage{fp}
%
%%%  Zur Darstellung des Euro-Symbols   %%
%\usepackage{eurosym, wasysym}
%\selectlanguage{german}
%
%%%   Fuer Bibtex nach APA Style (American Psychology Association)   %%
%\usepackage[numbers]{natbib}
\usebibitemtemplate{\insertbiblabel}

%% Code-Hervorhebung
%% Quellcode

%\usepackage[numbered,autolinebreaks,useliterate]{mcode}
%\usepackage{verbatim}            % Quellcode einbinden (\verbatiminput) standardpaket
%\usepackage{moreverb} 
%% PseudoCode
%\usepackage{algorithm}
\usepackage{algpseudocode}
%%\usepackage{algorithmicx}
%%\floatname{algorithm}{Algorithmus}
%\algrenewcommand{\algorithmiccomment}[1]{\hskip1em\textcolor{gray!60}{$\rhd$ #1}}
%%\renewcommand{\listalgorithmname}{Algorithmen}
%%\def\algorithmautorefname{Algorithmus}
%
%%% Code Highlighting
%\definecolor{mygray}{gray}{.75}
%\usepackage{listings} 
%\lstset{numbers=left, numberstyle=\tiny, numbersep=6pt} 
%\lstset{language=Python}
%\lstset{classoffset=1, morekeywords={mycontext}, keywordstyle=\color{darkgreen}, classoffset=0, keywordstyle=\color{darkblue}}
%\lstset{basicstyle=\small, showstringspaces=false, commentstyle=\color{mygray}, breaklines=true, captionpos=b}
%\renewcommand{\lstlistingname}{Code-Ausschnitt}
%\renewcommand{\lstlistlistingname}{Code-Ausschnitte}
%\def\lstlistingautorefname{Code-Ausschnitt}


%%%%%%%%%%%%%%%%%%%%%%%%%%%%%%%%%%%%%%%%%%%%%%%%%%%%%%%%%%%%%%%%%%%%%%%%%%%%%%%%%%%%%%%%%%%%
%%%                                   COMMAND SETUP                                       %%
%%%%%%%%%%%%%%%%%%%%%%%%%%%%%%%%%%%%%%%%%%%%%%%%%%%%%%%%%%%%%%%%%%%%%%%%%%%%%%%%%%%%%%%%%%%%

%#1 Breite
%#2 Datei (liegt im image Verzeichnis)
%#3 Beschriftung
%#4 Label fuer Referenzierung
\newcommand{\image}[4]{
\begin{figure}[H]
\centering
\includegraphics[width=#1]{#2}
\caption{\footnotesize{#3}}
\label{#4}
\end{figure}
}

% #1 videofile
% #2 scalefactor
\newcommand{\video}[2]{%
\includemovie[text={\includegraphics[scale=#2]{praesi/video/#1.png}}, autoplay, mouse=true, repeat=1]{}{}{praesi/video/#1.swf}}


\def\signed #1{{\leavevmode\unskip\nobreak\hfil\penalty50\hskip2em
  \hbox{}\nobreak\hfil(#1)%
  \parfillskip=0pt \finalhyphendemerits=0 \endgraf}}

\newsavebox\mybox
\newenvironment{aquote}[1]
  {\savebox\mybox{#1}\begin{fancyquotes}}
  {\signed{\usebox\mybox}\end{fancyquotes}}


\hyphenation{op-tical net-works semi-conduc-tor}

\setbeamertemplate{caption}[numbered]
%\numberwithin{figure}{section}
\begin{document}

\frame{\titlepage}

\AtBeginSection[]
{
  \frame<handout:0>
  {
    \frametitle{Outline}
    \begin{multicols}{2}
    \tableofcontents[currentsection,hideallsubsections]
    \end{multicols}
  }
}

%\AtBeginSubsection[]
%{
%  \frame<handout:0>
%  {
%    \frametitle{Outline}
%    \tableofcontents[sectionstyle=show/hide,subsectionstyle=show/shaded/hide,subsubsectionstyle=hide]
%  }
%}

%\AtBeginSubsubsection[]
%{
%  \frame<handout:0>
%  {
%    \frametitle{Outline}
%    \tableofcontents[sectionstyle=show/hide,subsectionstyle=show/shaded/hide,subsubsectionstyle=show/shaded/hide]
%  }
%}

\newcommand<>{\highlighton}[1]{%
  \alt#2{\structure{#1}}{{#1}}
}

\newcommand{\icon}[1]{\pgfimage[height=1em]{#1}}

\begin{frame}
\frametitle{\"Uber dieses Tutorium}
%\begin{block}{"Uber dieses Tutorium}
\begin{itemize}
\item Powered by Prof.\,Dr.\,Udo Krieger
\item In Version 10 überarbeitet und erweitert von Michael Träger und Valentin Barth (mit Unterstützung von Michael Timpelan)
\item Version 9 in \LaTeX -Beamer gesetzt und erweitert von Linus Dietz
\item Ursprünglicher Foliensatz von\begin{itemize}
\item Marcel Grossmann
\item Steffen Illig
\item Martin Sticht
\item Michael Timpelan
\end{itemize}
\item ca. 2600 Zeilen Code
\item Quelltext auf GitHub: \url{https://github.com/fs-wiai/LaTeX-Tutorial}
\end{itemize}
%\end{block}
\end{frame}

\frame<handout:0>
 {
%  \addtocounter{framenumber}{-1}
 	\frametitle{Zeitplan}
    \begin{itemize}
	    \item Mittagspause: ca. 12 Uhr (etwa 45 Minuten)
	    \item Ende: ca. 15:30 Uhr
	    \medskip
	    %TODO auch erwähnen?
	    \item[] Anschließend: Spezielle Fragen und individuelle Beratung %zu bereits von euch verfassten Dokumenten
    \end{itemize}
 }

 \frame<handout:0>
  {
 %  \addtocounter{framenumber}{-1}
  	\frametitle{Vorstellung}
     \begin{itemize}
 	    \item Wer sind wir?
 	    \begin{itemize}
        \item \textbf{Anika Amma} (Bachelor Angewandte Informatik)
        \item \textbf{Andreas Erhard} (Master Angewandte Informatik)
        \item \textbf{Florian Knoch} (Bachelor Software Systems Science)
        \item \textbf{Fabian Lamprecht} (Bachelor Angewandte Informatik)
        \item \textbf{Martin Müller} (Master Angewandte Informatik)
        \item \textbf{Johannes Rabold} (Master Angewandte Informatik)
        \item \textbf{Michael Träger} (Master Angewandte Informatik)
 	    \end{itemize}
 	    \medskip
 	    \item Wer seid ihr?
 	    \begin{itemize}
	 	    \item Fakultät
	 	    \item Studiengang
	 	    \item Semester
 	    \end{itemize}
     \end{itemize}
  }

\begin{frame}{\contentsname}
    \frametitle{Outline}
\begin{multicols}{2}
\tableofcontents[hideallsubsections]
\end{multicols}
\end{frame}


%%%%%%%%%%%%%%%%%%%%%%%%%%%%%%%%%%%%%%%%%
%%%%%%%%%% Content starts here %%%%%%%%%%
%%%%%%%%%%%%%%%%%%%%%%%%%%%%%%%%%%%%%%%%%
%\input{content/spielwiese}
\section{Intro} 

\subsection{Anwendungsbereiche, Sinn \& Zweck}
\begin{frame}[t]
\frametitle{Einf\"uhrung}
\framesubtitle{Sinn -- Unsinn -- Wahnsinn}
\bigskip
\bigskip
\bigskip

\begin{columns}[t]
\begin{column}{.3\textwidth}
\textbf{Sinnvoll}\\[3mm]
\begin{itemize}
\item Artikel
\item Bücher
\item wissenschaftliche Arbeiten
\item Bewerbungen
\end{itemize}
\end{column}
\begin{column}{.30\textwidth}
\textbf{Unsinn}\\[3mm]
\begin{itemize}
\item private Briefe
\item Geburtstags-einladungen
\item Getr"ankekarten
\end{itemize}
\end{column}
\begin{column}{.3\textwidth}
\textbf{Wahnsinn}\\[3mm]
\begin{itemize}
\item Einkaufszettel
\item Brainstorming
\item \ldots 
\end{itemize}
\end{column}
\end{columns}
\end{frame}

%-------------------------------------------------------------------------------

\begin{frame} 
\frametitle{Vom Code zum Dokument}
\framesubtitle{Kein WYSIWYG} 
\begin{columns}
\begin{column}{.7\textwidth}
\image{\textwidth}{image/worddoc.jpg}{\textbf{W}hat \textbf{Y}ou \textbf{S}ee \textbf{I}s \textbf{W}hat \textbf{Y}ou
\textbf{G}et}{img:worddoc}
\end{column}
\begin{column}{.3\textwidth}
\image{\textwidth}{image/codescreen.png}{\textbf{W}hat \textbf{W}ill \textbf{I} \textbf{G}et?}{img:codescreen}
%% Compile Animation
\end{column}
\end{columns}
\end{frame}

%-------------------------------------------------------------------------------

\begin{frame}
\frametitle{Einf\"uhrung}
\framesubtitle{Vorgehensweise}
\begin{columns}[onlytextwidth]
\begin{column}{0.40\textwidth}
\image{.8\textwidth}{image/codescreen.png}{Textdatei mit \LaTeX ~-Code}{img:code}
\end{column}
\begin{column}{0.25\textwidth}
\image{.8\textwidth}{image/miktex.jpg}{Compiler (z.B. MikTeX)}{img:miktex}
\end{column}
\begin{column}{0.25\textwidth}
\image{.6\textwidth}{image/pdflogo.png}{sch\"ones, lesbares und druckbares Dokument}{img:pdf}
\end{column}
\end{columns}
\end{frame}


%-------------------------------------------------------------------------------

\subsection{Vorteile \& Nachteile}
\begin{frame}
\frametitle{Einf\"uhrung}
\framesubtitle{Vorteile \& Nachteile}
Vorteile
\begin{itemize}
\item  dynamische Verzeichnisse und Referenzen
\item  automatische Layouts
\item  einfaches verteiltes Arbeiten möglich
\end{itemize}
Nachteile
\begin{itemize}
\item  Was kommt später raus?
\item  viele, zum Teil komplexe Befehle
\end{itemize}
\end{frame}

%-------------------------------------------------------------------------------

\subsection{\LaTeX --Compiler}
\begin{frame}
\frametitle{Einf\"uhrung}
\framesubtitle{\LaTeX - Compiler}
\begin{columns}[t]
\begin{column}{.4\textwidth}
\textbf{Software unter Windows:}\\
\begin{itemize}
  \item MikTex (http://www.miktex.org)\\
   2 Varianten: Basic oder  Complete
  %\item ProTeXt (http://www.tug.org/protext)\\
enthält MikTex, TeXnicCenter und Ghostscript – einfache Installation\\
\end{itemize}
\end{column}
\begin{column}{.6\textwidth}
\textbf{Software unter *nix:}
\begin{itemize}
  \item TeXLive\\
Pakete unter Ubuntu: {\ttfamily texlive-full} ist das Meta-Paket mit allen
ben\"otigten Paketen. Enthält auch Folgende:
\begin{itemize}
  \item {\ttfamily texlive-base
  \item texlive-lang-german}
\end{itemize}
Installation: {\ttfamily sudo apt-get install texlive-full}
\item MacOS: MacTeX (http://www.tug.org/mactex/2009)\\
\end{itemize}
\end{column}
\end{columns}
\end{frame}

%-------------------------------------------------------------------------------

%\subsection{Freie Editoren}
%
%\subsubsection{*nix}
%\begin{frame}
%\frametitle{Einf\"uhrung}
%\framesubtitle{Freie Editoren -- Linus \& MacOS }
%\begin{itemize}
% \item Kile\footnote{http://kile.sourceforge.net/}\\KDE-Programm, auch unter Gnome\slash Unity etc.
% verwendbar. Installation auf Debiansystemen mit {\ttfamily sudo apt-get
% install kile}.
%  \item Vim \LaTeX -suite (Plugin)\footnote{http://vim-latex.sourceforge.net/}\\
%  Ein Traum f"ur Vim-User.
%  \item TexShop (MacOS)\footnote{http://pages.uoregon.edu/koch/texshop/}
%\end{itemize}
%\end{frame}

%-------------------------------------------------------------------------------

%\subsubsection{Windows}
%\begin{frame}
%\frametitle{Einf\"uhrung}
%\framesubtitle{Freie Editoren -- Windows}
%\begin{itemize}
%\item TeXnicCenter\footnote{http://www.texniccenter.org/}
%  %\item \ldots
%\end{itemize}
%\end{frame}

%-------------------------------------------------------------------------------

\subsubsection{Cross-Platform}
\begin{frame}
\frametitle{Einf\"uhrung}
\framesubtitle{Freie Editoren -- Cross-Platform}
\begin{itemize}
  \item TeXMaker\footnote{http://www.xm1math.net/texmaker}\\
   Sehr solide, verwenden wir hier im Tutorium.
  \item TeXstudio\footnote{http://sourceforge.net/projects/texstudio/?source=dlp}\\
  "Ahnlich wie der TeXMaker, allerdings etwas m"achtiger.
  \item TeXlipse\footnote{http://texlipse.sourceforge.net/}\\ F"ur fortgeschrittene User, Plugin f"ur
  Eclipse. Gute IDE-Unterst\"utzung, Code-Completion, Autobuilds, Versionsverwaltung etc.
\end{itemize}
\end{frame}

%-------------------------------------------------------------------------------

%TODO Online: Sharelatex, Overleaf (siehe Timpelan - FIM Mannheim)

%TODO TexStudio

%-------------------------------------------------------------------------------

\begin{frame}
\frametitle{TeXmaker}
\framesubtitle{\"Uberblick}
\image{\textwidth}{image/texmaker_overview.png}{Das Standardfenster des Texmaker}{img:texmaker1}

\end{frame}

%-------------------------------------------------------------------------------

%\begin{frame}
%\frametitle{TeXmaker}
%\framesubtitle{Synctex}
%\image{\textwidth}{image/synctex.png}{Synctex}{img:synctex}

%\end{frame}

\begin{frame}
\frametitle{Mein erstes \LaTeX~-Dokument}
\begin{block}{Neue Befehle:}
\begin{itemize}
\item \begin{ttfamily}\color{nounibaredII}\textbackslash documentclass\color{nounibagreenI}\color{black}\{article\}\end{ttfamily}
\item \begin{ttfamily}\color{unibablueI}\textbackslash begin\color{black}\{document\}\end{ttfamily}
\item \begin{ttfamily} Inhalt als plain text \end{ttfamily}
\item \begin{ttfamily}\color{unibablueI}\textbackslash end\color{black}\{document\}\end{ttfamily}
\end{itemize}
\end{block}
Das ist alles was man f\"ur ein \LaTeX -Dokument braucht. Und das probieren wir jetzt aus!

\end{frame}

\section{$\mathcal{A}1$}
\begin{frame}
\frametitle{Aufgabe 1}
\framesubtitle{Schreibt und kompiliert ``Hello World!''}
\begin{center}
\begin{rm}
\Large Hello World!\\
\end{rm}
\end{center}
\bigskip
\begin{block}{Aufgabe 1}
\begin{itemize}
\item Ladet aus dem Virtuellen Campus (\url{https://wiai.de/latex}) die Verzeichnisvorlage herunter und entpackt sie!
\item Schreibt und kompiliert \glqq{}Hello World\grqq! Ihr könnt direkt im entpackten Ordner \texttt{OrdnerStrukturVorgabe} arbeiten.
\item \textit{Hinweis:} Normale \LaTeX -Dateien haben {\ttfamily .tex} als Dateiendung
\end{itemize}
\end{block}
\begin{alertblock}{\textbf{Achtung:}}
Die Computer im PC-Pool werden automatisch zurückgesetzt. Speichert eure Daten \textbf{auf dem Heimlaufwerk \texttt{W://}}, um sie zu behalten.
\end{alertblock}
\end{frame}

\section{Formatierung}

\begin{frame}
\frametitle{Ein erstes Anwendungsbeispiel}
\framesubtitle{\"Uberschriften, Inhaltsverzeichnis, einfache Formatierung,
Sonderzeichen}
\begin{block}{Neue Befehle in diesem Abschnitt}
\begin{multicols}{2}
\begin{itemize}
  \item \begin{ttfamily}\color{nounibaredII}\textbackslash usepackage\color{black}\{Paket\}
  \item \color{nounibaredII}\textbackslash befehl\color{nounibagreenI}[evtl\_optionen]\color{black}\{\\Formatierter\_Text\}
  \item \color{unibablueI}\textbackslash begin\color{black}\{Umgebung\}
  \item \color{unibablueI}\textbackslash end\color{black}\{Umgebung\}
  \item \color{nounibaredII}$\backslash\backslash$\color{black}
  \item \color{nounibaredI}\textbackslash newpage\color{black}
  \item \color{unibablueI}\textbackslash sub$^*$section\color{black}\{Titel\}
  \item $\color{nounibaredII}\backslash$\color{nounibaredII}textbf\color{black}\{Text\}
  \item $\color{nounibaredII}\backslash$\color{nounibaredII}textit\color{black}\{Text\}
  \item $\color{nounibaredII}\backslash$\color{nounibaredII}underline\color{black}\{Text\}
  \item \color{nounibaredI}$\color{nounibaredI}\backslash$tiny
  \item \color{nounibaredI}$\color{nounibaredI}\backslash$scriptsize
  \item \color{nounibaredI}$\color{nounibaredI}\backslash$footnotesize
  \item \color{nounibaredI}$\color{nounibaredI}\backslash$normalsize
  \item \color{nounibaredI}$\color{nounibaredI}\backslash$large
  \item \color{nounibaredI}$\color{nounibaredI}\backslash$Large
  \item \color{nounibaredI}$\color{nounibaredI}\backslash$LARGE
  \item \color{nounibaredI}$\color{nounibaredI}\backslash$huge\end{ttfamily}
\end{itemize}
\end{multicols}
\end{block}
\end{frame}


%\begin{frame}
%\frametitle{Ein erstes Anwendungsbeispiel}
%\framesubtitle{Pakete einbinden und Befehle anwenden}
%\begin{itemize}
%  \item Pakete sind Sammlungen von Befehlen oder enthalten z.B. Zeichensätze.\\ Sie werden zu
%  Beginn einer \TeX-Datei angegeben:\\
%  \smallskip
%\textbf{\begin{ttfamily}\color{nounibaredII}\textbackslash usepackage\color{black}\{babel\}
%
%\smallskip
%\end{ttfamily}}
% Einbinden des Paketes „\begin{ttfamily}babel\end{ttfamily}“. (F\"ur Internationalisierung)
%\item Schreibweise von Latex-Befehlen:
%
%\textbf{\begin{ttfamily}\color{nounibaredII}\textbackslash befehl\color{nounibagreenI}[evtl\_optionen]\color{black}\{Formatierter\_Text\}\end{ttfamily}}
%\begin{itemize}
%  \item in \begin{ttfamily}\{\}\end{ttfamily} stehen immer notwendige Parameter bzw. Text
% \item in \begin{ttfamily}[ ]\end{ttfamily} stehen (falls vorhanden)
% zus"atzliche, optionale Parameter
% \item zum Beispiel:
%
%
%\begin{ttfamily}
%\color{nounibaredII}\textbackslash documentclass\color{nounibagreenI}[a4paper,12pt,pdftex,ngerman]\color{black}\{article\}
%\end{ttfamily}
%\end{itemize}
%\end{frame}

\begin{frame}
\frametitle{Ein erstes Anwendungsbeispiel}
\framesubtitle{Befehle cont't}
\begin{columns}
\begin{column}{0.6\textwidth}
\begin{ttfamily}\scriptsize
\color{nounibaredI}\color{nounibaredI}\textbackslash documentclass\color{black}\color{nounibagreenI}[a4paper, pdftex, ngerman]\color{black}\{article\} \\
\color{nounibaredI}\color{nounibaredI}\textbackslash usepackage\color{black}\color{nounibagreenI}[utf8]\color{black}\{inputenc\} \\
\color{nounibaredI}\color{nounibaredI}\textbackslash usepackage\color{black}\color{nounibagreenI}[T1]\color{black}\{fontenc\} \\
\color{nounibaredI}\color{nounibaredI}\textbackslash usepackage\color{black}\{babel\} \\
\color{nounibaredI}\color{unibablueI}\textbackslash\color{unibablueI}begin\color{black}\color{black}\{document\} \\
Das ist ein einfaches Minidokument \\
ohne Besonderheiten. Zeilenumbrüche \\
funktionieren immer automatisch! \\
Mehrere \\
Leerzeichen hintereinander werden  \\
zu einem zusammengefasst. \\
Getrennt wird auch automatisch.\color{nounibaredI}\color{nounibaredI}\textbackslash \color{nounibaredI}\textbackslash \color{black} \\
Mit zwei Backslashs beginnt eine neue \\
Zeile.\color{nounibaredI}\color{nounibaredI}\textbackslash \color{nounibaredI}\textbackslash \color{black} \\
Ein neuer Absatz entsteht durch eine \\
leere Zeile. \\
\color{nounibaredI}\color{unibablueI}\textbackslash\color{unibablueI}end\color{black}\color{black}\{document\} \\

\end{ttfamily}
\end{column}

\begin{column}{0.4\textwidth}
Es gibt verschiedene Arten von Dokumenten.\\ Hier wird die Dokumentenart
\begin{ttfamily}article\end{ttfamily} verwendet (weiter möglich:
\begin{ttfamily}book\end{ttfamily} und \begin{ttfamily}report\end{ttfamily}) In
\begin{ttfamily}[]\end{ttfamily} steht die Papiergröße und die Schriftgröße des
Standardtextes.\\
%! TODO!
\end{column}
\end{columns}
\end{frame}

\begin{frame}
\frametitle{Ein erstes Anwendungsbeispiel}
\framesubtitle{Befehle cont't}
\begin{columns}
\begin{column}{0.6\textwidth}
\begin{ttfamily}\scriptsize
\color{nounibaredI}\color{nounibaredI}\textbackslash documentclass\color{black}\color{nounibagreenI}[a4paper, pdftex, ngerman]\color{black}\{article\} \\
\color{nounibaredI}\color{nounibaredI}\textbackslash usepackage\color{black}\color{nounibagreenI}[utf8]\color{black}\{inputenc\} \\
\color{nounibaredI}\color{nounibaredI}\textbackslash usepackage\color{black}\color{nounibagreenI}[T1]\color{black}\{fontenc\} \\
\color{nounibaredI}\color{nounibaredI}\textbackslash usepackage\color{black}\{babel\} \\
\color{nounibaredI}\color{unibablueI}\textbackslash\color{unibablueI}begin\color{black}\color{black}\{document\} \\
Das ist ein einfaches Minidokument \\
ohne Besonderheiten. Zeilenumbrüche \\
funktionieren immer automatisch! \\
Mehrere \\
Leerzeichen hintereinander werden  \\
zu einem zusammengefasst. \\
Getrennt wird auch automatisch.\color{nounibaredI}\color{nounibaredI}\textbackslash \color{nounibaredI}\textbackslash \color{black} \\
Mit zwei Backslashs beginnt eine neue \\
Zeile.\color{nounibaredI}\color{nounibaredI}\textbackslash \color{nounibaredI}\textbackslash \color{black} \\
Ein neuer Absatz entsteht durch eine \\
leere Zeile. \\
\color{nounibaredI}\color{unibablueI}\textbackslash\color{unibablueI}end\color{black}\color{black}\{document\} \\

 \normalsize
\end{ttfamily}
\end{column}
\begin{column}{0.4\textwidth}
\begin{ttfamily}\textbf{\color{unibablueI}\textbackslash begin\color{black}\{Umgebung\}}\end{ttfamily}\\
Es beginnt eine neue Umgebung, hier das eigentliche Dokument.\\[5mm]

\begin{ttfamily}\textbf{\color{unibablueI}\textbackslash end\color{black}\{Umgebung\}}\end{ttfamily}\\
Die mit \begin{ttfamily}\textbf{\color{unibablueI}\textbackslash begin}\color{black}\{\}\end{ttfamily}
eingeleitete Umgebung ist hier zu Ende.\\[5mm]

\begin{ttfamily}\textbf{\color{nounibaredII}$\backslash\backslash$}\color{black}
~Zeilenumbruch\end{ttfamily}\\
\end{column}
\end{columns}
\end{frame}



\begin{frame}
\frametitle{Ein erstes Anwendungsbeispiel}
\framesubtitle{Pakete}
\begin{columns}
\begin{column}{0.6\textwidth}
\begin{ttfamily}\scriptsize
\color{nounibaredI}\color{nounibaredI}\textbackslash documentclass\color{black}\color{nounibagreenI}[a4paper, pdftex, ngerman]\color{black}\{article\} \\
\color{nounibaredI}\color{nounibaredI}\textbackslash usepackage\color{black}\color{nounibagreenI}[utf8]\color{black}\{inputenc\} \\
\color{nounibaredI}\color{nounibaredI}\textbackslash usepackage\color{black}\color{nounibagreenI}[T1]\color{black}\{fontenc\} \\
\color{nounibaredI}\color{nounibaredI}\textbackslash usepackage\color{black}\{babel\} \\
\color{nounibaredI}\color{unibablueI}\textbackslash\color{unibablueI}begin\color{black}\color{black}\{document\} \\
Das ist ein einfaches Minidokument \\
ohne Besonderheiten. Zeilenumbrüche \\
funktionieren immer automatisch! \\
Mehrere \\
Leerzeichen hintereinander werden  \\
zu einem zusammengefasst. \\
Getrennt wird auch automatisch.\color{nounibaredI}\color{nounibaredI}\textbackslash \color{nounibaredI}\textbackslash \color{black} \\
Mit zwei Backslashs beginnt eine neue \\
Zeile.\color{nounibaredI}\color{nounibaredI}\textbackslash \color{nounibaredI}\textbackslash \color{black} \\
Ein neuer Absatz entsteht durch eine \\
leere Zeile. \\
\color{nounibaredI}\color{unibablueI}\textbackslash\color{unibablueI}end\color{black}\color{black}\{document\} \\

 \normalsize
\end{ttfamily}
\end{column}
\begin{column}{0.4\textwidth}
\begin{ttfamily}\textbf{ngerman}\end{ttfamily}\\
Für Deutschland typische Formatierungen und (Trenn)-Regeln werden verwendet.\\[5mm]

\begin{ttfamily}\textbf{inputenc}\end{ttfamily}\\
Definiert den Zeichen-\\satz, der verwendet werden soll. Es sollte immer
\begin{ttfamily}UTF-8\end{ttfamily} verwendet werden, weil er universal auf
allen Betriebssystemen l"auft.\\
\end{column}
\end{columns}
\end{frame}

\begin{frame}
\frametitle{Exkurs}
\framesubtitle{Zeichenkodierungen}
\begin{columns}
\begin{column}{0.6\textwidth}
\image{\textwidth}{image/utf8.png}{UTF-8 im Texmaker}{img:utf8}

\end{column}
\begin{column}{0.4\textwidth}
Wird ein Dokument geöffnet, wird automatisch der richtige Zeichensatz benutzt.
Beim Erstellen neuer Dokumente wird die Datei in dem Format gespeichert, die im
 Editor voreingestellt ist. In den Texmaker-Einstelllungen muss derselbe Zeichensatz verwendet werden, der
auch im erstellten LaTeX-Dokument verwendet wird.\\
\end{column}
\end{columns}
\textbf{Bei Gruppenarbeiten muss jedes Mitglied zwingend \underline{UTF-8} im
Editor einstellen, sonst ist Ärger so gut wie vorprogrammiert!} (Kaputte
Umlaute, Kompilierungsfehler uvm., wenn es nicht nur „Windows“-User gibt.)
\end{frame}

\begin{frame}
\frametitle{Ein erstes Anwendungsbeispiel}
\framesubtitle{Als .PDF}
\begin{columns}
\begin{column}{0.5\textwidth}
\begin{ttfamily}\scriptsize
\color{nounibaredI}\color{nounibaredI}\textbackslash documentclass\color{black}\color{nounibagreenI}[a4paper, pdftex, ngerman]\color{black}\{article\} \\
\color{nounibaredI}\color{nounibaredI}\textbackslash usepackage\color{black}\color{nounibagreenI}[utf8]\color{black}\{inputenc\} \\
\color{nounibaredI}\color{nounibaredI}\textbackslash usepackage\color{black}\color{nounibagreenI}[T1]\color{black}\{fontenc\} \\
\color{nounibaredI}\color{nounibaredI}\textbackslash usepackage\color{black}\{babel\} \\
\color{nounibaredI}\color{unibablueI}\textbackslash\color{unibablueI}begin\color{black}\color{black}\{document\} \\
Das ist ein einfaches Minidokument \\
ohne Besonderheiten. Zeilenumbrüche \\
funktionieren immer automatisch! \\
Mehrere \\
Leerzeichen hintereinander werden  \\
zu einem zusammengefasst. \\
Getrennt wird auch automatisch.\color{nounibaredI}\color{nounibaredI}\textbackslash \color{nounibaredI}\textbackslash \color{black} \\
Mit zwei Backslashs beginnt eine neue \\
Zeile.\color{nounibaredI}\color{nounibaredI}\textbackslash \color{nounibaredI}\textbackslash \color{black} \\
Ein neuer Absatz entsteht durch eine \\
leere Zeile. \\
\color{nounibaredI}\color{unibablueI}\textbackslash\color{unibablueI}end\color{black}\color{black}\{document\} \\

 \normalsize
\end{ttfamily}
\end{column}

\begin{column}{0.5\textwidth}
\image{\textwidth}{image/minidocument.png}{Der Code von der linken Seite als .pdf.}{listing:minidocument}
\end{column}
\end{columns}
\end{frame}



\begin{frame}
\frametitle{Abschnitte}
\framesubtitle{Kapitelmarken}
\begin{columns}
\begin{column}{0.5\textwidth}
\begin{ttfamily}\scriptsize
\color{nounibaredI}\color{nounibaredI}\textbackslash documentclass\color{black}\color{nounibagreenI}[a4paper, pdftex, 12pt, ngerman]\color{black}\{article\} \\
\color{nounibaredI}\color{nounibaredI}\textbackslash usepackage\color{black}\color{nounibagreenI}[utf8]\color{black}\{inputenc\} \\
\color{nounibaredI}\color{nounibaredI}\textbackslash usepackage\color{black}\color{nounibagreenI}[T1]\color{black}\{fontenc\} \\
\color{nounibaredI}\color{nounibaredI}\textbackslash usepackage\color{black}\{babel\} \\
\color{nounibaredI}\color{unibablueI}\textbackslash\color{unibablueI}begin\color{black}\color{black}\{document\} \\
\color{nounibaredI}\color{nounibaredI}\textbackslash tableofcontents\color{black} \\
\color{nounibaredI}\color{nounibaredI}\textbackslash newpage\color{black} \\
\color{nounibaredI}\color{unibablueI}\textbackslash\color{unibablueI}section\color{black}\color{black}\{Kapitel 1\} \\
Hier kommt der erste Teil. \\
\color{nounibaredI}\color{unibablueI}\textbackslash\color{unibablueI}subsection\color{black}\color{black}\{Unterkapitel 1\} \\
Das erste Unterkapitel. \\
\color{nounibaredI}\color{unibablueI}\textbackslash\color{unibablueI}subsection\color{black}\color{black}\{Unterkapitel 2\} \\
Und noch ein Unterkapitel. \\
\color{nounibaredI}\color{unibablueI}\textbackslash\color{unibablueI}subsubsection\color{black}\color{black}\{Unterunterkapitel 1\} \\
Das ist ein Unterkapitel von einem Unterkapitel. \\
\color{nounibaredI}\color{unibablueI}\textbackslash\color{unibablueI}end\color{black}\color{black}\{document\} \\

\end{ttfamily}
\end{column}
\begin{column}{0.5\textwidth}
\begin{ttfamily}\color{nounibaredI}\textbackslash newpage\color{black}\end{ttfamily}\\
Seitenumbruch\\[3mm]
\begin{ttfamily}\color{unibablueI}\textbackslash section\color{black}\{Titel\}\end{ttfamily}\\
Ein neuer Abschnitt mit dem in \begin{ttfamily}\{\}\end{ttfamily} angegebenen Titel
beginnt.\\[3mm]
\begin{ttfamily}\color{unibablueI}\textbackslash subsection\color{black}\{Titel\}\end{ttfamily}\\
Ein Unterabschnitt.\\[3mm]
\begin{ttfamily}\color{unibablueI}\textbackslash subsubsection\color{black}\{Titel\}\end{ttfamily}\\
Noch eine Ebene darunter.\\
\end{column}
\end{columns}
\end{frame}

\begin{frame}
\frametitle{Abschnitte}
\framesubtitle{Kapitelmarken .PDF}
\begin{columns}
\begin{column}{0.45\textwidth}
\begin{ttfamily}\scriptsize
\color{nounibaredI}\color{nounibaredI}\textbackslash documentclass\color{black}\color{nounibagreenI}[a4paper, pdftex, 12pt, ngerman]\color{black}\{article\} \\
\color{nounibaredI}\color{nounibaredI}\textbackslash usepackage\color{black}\color{nounibagreenI}[utf8]\color{black}\{inputenc\} \\
\color{nounibaredI}\color{nounibaredI}\textbackslash usepackage\color{black}\color{nounibagreenI}[T1]\color{black}\{fontenc\} \\
\color{nounibaredI}\color{nounibaredI}\textbackslash usepackage\color{black}\{babel\} \\
\color{nounibaredI}\color{unibablueI}\textbackslash\color{unibablueI}begin\color{black}\color{black}\{document\} \\
\color{nounibaredI}\color{nounibaredI}\textbackslash tableofcontents\color{black} \\
\color{nounibaredI}\color{nounibaredI}\textbackslash newpage\color{black} \\
\color{nounibaredI}\color{unibablueI}\textbackslash\color{unibablueI}section\color{black}\color{black}\{Kapitel 1\} \\
Hier kommt der erste Teil. \\
\color{nounibaredI}\color{unibablueI}\textbackslash\color{unibablueI}subsection\color{black}\color{black}\{Unterkapitel 1\} \\
Das erste Unterkapitel. \\
\color{nounibaredI}\color{unibablueI}\textbackslash\color{unibablueI}subsection\color{black}\color{black}\{Unterkapitel 2\} \\
Und noch ein Unterkapitel. \\
\color{nounibaredI}\color{unibablueI}\textbackslash\color{unibablueI}subsubsection\color{black}\color{black}\{Unterunterkapitel 1\} \\
Das ist ein Unterkapitel von einem Unterkapitel. \\
\color{nounibaredI}\color{unibablueI}\textbackslash\color{unibablueI}end\color{black}\color{black}\{document\} \\

\end{ttfamily}
\end{column}
\begin{column}{0.55\textwidth}
\image{\textwidth}{image/chapters.png}{Die Kapitel werden automatisch mitgez\"ahlt}{img:chapters}
\end{column}
\end{columns}
\end{frame}


\begin{frame}
\frametitle{Abschnitte}
\framesubtitle{Part \& Chapter}
Neben \begin{ttfamily}\color{unibablueI}\textbackslash section\color{black}\{\},
\color{unibablueI}\textbackslash subsection\color{black}\{\}\end{ttfamily}, und \begin{ttfamily}\color{unibablueI}\textbackslash subsubsection\color{black}\{\}\end{ttfamily} gibt es auch noch den Befehl
 \begin{ttfamily}\color{unibablueI}\textbackslash part\color{black}\{\}\end{ttfamily} welcher einen größeren Teil definiert.
\begin{ttfamily}\color{unibablueI}\textbackslash part\color{black}\{\}\end{ttfamily} füllt eine ganze eigene Seite.\\
Neben dem Dokumentypen {\ttfamily article} existieren f"ur Flie\ss
textdokumente noch weitere wie {\ttfamily book} und {\ttfamily report}.\\
Bei {\ttfamily book} wird in der Regel zwischen linker und rechter Seite
unterschieden, wobei die sich z.B. darin unterscheiden, ob die Seitenzahl links oder rechts steht, bzw. was sonst noch in der Kopf- oder Fußzeile stehen kann.
In {\ttfamily book} und {\ttfamily report} gibt es noch den
Gliederungsbefehl \begin{ttfamily}\color{unibablueI}\textbackslash chapter\color{black}\{\}\end{ttfamily}.

%\begin{columns}
%\begin{column}{0.5\textwidth}
%CODE
%\end{column}
%\begin{column}{0.5\textwidth}
%OUTPUT
%\end{column}
%\end{columns}
\end{frame}


\begin{frame}
\frametitle{Formatierungen}
\framesubtitle{FettKursivUnterstrichen}
\begin{columns}
\begin{column}{0.45\textwidth}
\begin{ttfamily}\scriptsize
\color{nounibaredI}\color{nounibaredI}\textbackslash documentclass\color{black}\color{nounibagreenI}[a4paper, pdftex, 12pt, ngerman]\color{black}\{article\} \\
\color{nounibaredI}\color{nounibaredI}\textbackslash usepackage\color{black}\color{nounibagreenI}[utf8]\color{black}\{inputenc\} \\
\color{nounibaredI}\color{nounibaredI}\textbackslash usepackage\color{black}\color{nounibagreenI}[T1]\color{black}\{fontenc\} \\
\color{nounibaredI}\color{unibablueI}\textbackslash\color{unibablueI}begin\color{black}\color{black}\{document\} \\
Among others there are the following options:\color{nounibaredI}\color{nounibaredI}\textbackslash \color{nounibaredI}\textbackslash \color{black} \\
\color{nounibaredI}\color{nounibaredI}\textbackslash textbf\color{black}\{bold\}\color{nounibaredI}\color{nounibaredI}\textbackslash \color{nounibaredI}\textbackslash \color{black} \\
\color{nounibaredI}\color{nounibaredI}\textbackslash textit\color{black}\{italic\}\color{nounibaredI}\color{nounibaredI}\textbackslash \color{nounibaredI}\textbackslash \color{black} \\
\color{nounibaredI}\color{nounibaredI}\textbackslash underline\color{black}\{underlined\}\color{nounibaredI}\color{nounibaredI}\textbackslash \color{nounibaredI}\textbackslash \color{black} \\
\color{nounibaredI}\color{nounibaredI}\textbackslash underline\color{black}\{\color{nounibaredI}\color{nounibaredI}\textbackslash textbf\color{black}\{underlined and bold\}\}\color{nounibaredI}\color{nounibaredI}\textbackslash \color{nounibaredI}\textbackslash \color{black} \\
\color{nounibaredI}\color{unibablueI}\textbackslash\color{unibablueI}end\color{black}\color{black}\{document\} \\

\end{ttfamily}
\end{column}
\begin{column}{0.55\textwidth}
Unter anderem folgende M"oglichkeiten:\\[3mm]
\textbf{fetter}\\
\textit{kursiver}\\
\underline{unterstrichener}\\
\underline{\textbf{unterstrichen und fett}}
%\input{formats_pdf.tex}
\begin{block}{Textformatierungen}
\begin{ttfamily}$\color{nounibaredII}\backslash$\color{nounibaredII}textbf\color{black}\{Text\}\end{ttfamily}
fetter Text\\
\begin{ttfamily}$\color{nounibaredII}\backslash$\color{nounibaredII}textit\color{black}\{Text\}\end{ttfamily}
kursiver Text\\
\begin{ttfamily}$\color{nounibaredII}\backslash$\color{nounibaredII}underline\color{black}\{Text\}\end{ttfamily}
unterstrichen
\end{block}
\end{column}
\end{columns}
\end{frame}

\begin{frame}
\frametitle{Formatierungen}
\framesubtitle{Schriftgr"o\ss e}
\begin{columns}
\begin{column}{0.5\textwidth}
\begin{ttfamily}\scriptsize
\color{nounibaredI}\color{nounibaredI}\textbackslash documentclass\color{black}\color{nounibagreenI}[a4paper, pdftex, 12pt,ngerman]\color{black}\{article\} \\
\color{nounibaredI}\color{nounibaredI}\textbackslash usepackage\color{black}\color{nounibagreenI}[utf8]\color{black}\{inputenc\} \\
\color{nounibaredI}\color{nounibaredI}\textbackslash usepackage\color{black}\color{nounibagreenI}[T1]\color{black}\{fontenc\} \\
\color{nounibaredI}\color{nounibaredI}\textbackslash usepackage\color{black}\{babel\} \\
\color{nounibaredI}\color{unibablueI}\textbackslash\color{unibablueI}begin\color{black}\color{black}\{document\} \\
\color{nounibaredI}\color{nounibaredI}\textbackslash tiny \color{black} unlesbarer Text \color{nounibaredI}\color{nounibaredI}\textbackslash \color{nounibaredI}\textbackslash \color{black} \\
\color{nounibaredI}\color{nounibaredI}\textbackslash scriptsize \color{black} sehr kleiner Text\color{nounibaredI}\color{nounibaredI}\textbackslash \color{nounibaredI}\textbackslash \color{black} \\
\color{nounibaredI}\color{nounibaredI}\textbackslash footnotesize \color{black} Fussnotengröße \color{nounibaredI}\color{nounibaredI}\textbackslash \color{nounibaredI}\textbackslash \color{black} \\
\color{nounibaredI}\color{nounibaredI}\textbackslash normalsize \color{black} Standartgröße \color{nounibaredI}\color{nounibaredI}\textbackslash \color{nounibaredI}\textbackslash \color{black} \\
\color{nounibaredI}\color{nounibaredI}\textbackslash large \color{black} größer\color{nounibaredI}\color{nounibaredI}\textbackslash \color{nounibaredI}\textbackslash \color{black} \\
\color{nounibaredI}\color{nounibaredI}\textbackslash Large \color{black} noch größer \color{nounibaredI}\color{nounibaredI}\textbackslash \color{nounibaredI}\textbackslash \color{black} \\
\color{nounibaredI}\color{nounibaredI}\textbackslash LARGE \color{black} sehr Groß \color{nounibaredI}\color{nounibaredI}\textbackslash \color{nounibaredI}\textbackslash \color{black} \\
\color{nounibaredI}\color{nounibaredI}\textbackslash huge \color{black} riesig \color{nounibaredI}\color{nounibaredI}\textbackslash \color{nounibaredI}\textbackslash \color{black} \\
\color{nounibaredI}\color{unibablueI}\textbackslash\color{unibablueI}end\color{black}\color{black}\{document\} \\

\end{ttfamily}
\end{column}
\begin{column}{0.5\textwidth}
\rm \tiny ~ein wenig Text \\
\scriptsize ~ein wenig Text \\
\normalsize ~ein wenig Text \\
\large ~ein wenig Text \\
\Large ~ein wenig Text \\
\LARGE ~ein wenig Text \\
\huge  ~ein wenig Text \\
\end{column}
\end{columns}
\end{frame}

\begin{frame}
\frametitle{Formatierungen}
\framesubtitle{FettKursivunterstrichen}
\begin{columns}
\begin{column}{0.5\textwidth}
\begin{ttfamily}\scriptsize\color{nounibaredII}\textbackslash documentclass\color{nounibagreenI}[a4paper, pdftex, 12pt, ngerman]\color{black}\{article\}\\[3mm] 
$\color{nounibaredII}\backslash$\color{nounibaredII}usepackage\color{nounibagreenI}[utf8]\color{black}\{inputenc\}\\
$\color{nounibaredII}\backslash$\color{nounibaredII}usepackage\color{nounibagreenI}[T1]\color{black}\{fontenc\}\\
$\color{nounibaredII}\backslash$\color{nounibaredII}usepackage\color{nounibagreenI}[iso]\color{black}\{umlaute\}\\
$\color{nounibaredII}\backslash$\color{nounibaredII}usepackage\color{black}\{babel\}\\
\color{gray}\% NEU NEU NEU\\
$\color{nounibaredII}\backslash$\color{nounibaredII}usepackage\color{black}\{eurosym\}\\
$\color{unibablueI}\backslash$\color{unibablueI}begin\color{black}\{document\}\\
$\color{nounibaredII}\backslash$\color{nounibaredII}textit\color{black}\{Einige
Sonderzeichen:\}\\
\color{nounibaredII}\textbackslash \% \textbackslash \$ \textbackslash \& \textbackslash \{ \textbackslash \}
\textbackslash \_ \textbackslash \# \textbackslash S \textbackslash copyright\\
\textbackslash slash \~ ~ \color{unibayellowI}\$\color{nounibaredII}$\color{nounibaredII}\backslash$backslash\color{unibayellowI}\$\color{nounibaredII}  ~\textbackslash euro \\

$\color{nounibaredII}\backslash$\color{nounibaredII}textit\color{black}\{Binde-\slash
Gedanken-\slash Trennstriche:\} \\
- -- --- \color{unibayellowI}\$\color{black}-\color{unibayellowI}\$\color{black} (letzteres mathematisches Minus) \\

$\color{nounibaredII}\backslash$\color{nounibaredII}textit\color{black}\{Anf"uhrungszeichen aus \begin{ttfamily}ngerman\end{ttfamily}:\} \\
\color{nounibaredII}\textbackslash glqq \textbackslash grqq \textbackslash flqq \textbackslash frqq\\
\color{unibablueI}\textbackslash end\color{black}\{document\}
\end{ttfamily}
\end{column}
\begin{column}{0.5\textwidth}
\textit{Einige Sonderzeichen:}    \\
\% \$ \& \{ \} \_ \# \S ~ \copyright \slash ~ \textbackslash  \euro \\

\textit{Binde-\slash Gedanken-\slash Trennstriche:} \\
- -- --- $-$ (letzteres mathematisches Minus) \\

\textit{Anführungszeichen aus (n)german:} \\
\glqq \grqq \flqq \frqq\\[5mm]
Für das \euro -Zeichen wird das Package \begin{ttfamily}eurosym\end{ttfamily}
benötigt.\\

\end{column}
\end{columns}
\medskip
\footnotesize Sonderzeichen müssen mit dem '\color{nounibaredI}\textbackslash \color{black}' eingeführt werden.
Manchmal, z.B. in \"Uberschriften m\"ussen Umlaute des Pakets ngerman mit \grqq a \grqq o
\grqq u und das \ss ~mit \color{nounibaredI} \textbackslash ss \color{black}gebildet werden, ansonsten reicht es das Packet {\ttfamily babel} einzubinden.
\end{frame}

\section{Fu\ss noten}

\begin{frame}
\frametitle{Fu\ss noten}
\framesubtitle{Einbau von Fu\ss noten}
\begin{block}{Neue Befehle in diesem Abschnitt}
\begin{itemize}
\item \color{nounibaredI}\textbackslash footnote
\item \textbackslash footnotemark\color{nounibagreenI}[]\color{black}
\end{itemize}
\end{block}
\end{frame}

%-------------------------------------------------------------------------------

\begin{frame}
\frametitle{Fu\ss noten}
\framesubtitle{Befehle}
\begin{columns}
\begin{column}{0.5\textwidth}
\begin{ttfamily}\scriptsize
\color{nounibaredI}\color{nounibaredI}\textbackslash documentclass\color{black}\color{nounibagreenI}[a4paper, pdftex, 12pt, ngerman]\color{black}\{article\} \\
\color{nounibaredI}\color{nounibaredI}\textbackslash usepackage\color{black}\color{nounibagreenI}[utf8]\color{black}\{inputenc\} \\
\color{nounibaredI}\color{nounibaredI}\textbackslash usepackage\color{black}\color{nounibagreenI}[T1]\color{black}\{fontenc\} \\
\color{nounibaredI}\color{nounibaredI}\textbackslash usepackage\color{black}\{babel\} \\
\color{nounibaredI}\color{unibablueI}\textbackslash\color{unibablueI}begin\color{black}\color{black}\{document\} \\
The footnote\color{nounibaredI}\color{nounibaredI}\textbackslash footnote\color{black}\{here comes the footnote text\} to a word or a text always appears on the page it belongs to. The footnote text is bracketed.\color{nounibaredI}\color{nounibaredI}\textbackslash \color{nounibaredI}\textbackslash \color{black} \\
\color{nounibaredI}\color{nounibaredI}\textbackslash \color{nounibaredI}\textbackslash \color{black} \\
A manual numbering is possible, too.\color{nounibaredI}\color{nounibaredI}\textbackslash footnote\color{black}\color{nounibagreenI}[10]\color{black}\{just as this\}, even without footnotetext\color{nounibaredI}\color{nounibaredI}\textbackslash footnotemark\color{black}\color{nounibagreenI}[2]\color{black}. \\
\color{nounibaredI}\color{unibablueI}\textbackslash\color{unibablueI}end\color{black}\color{black}\{document\} \\

\end{ttfamily}
\end{column}
\begin{column}{0.5\textwidth}
\begin{ttfamily}\color{nounibaredI}\textbackslash footnote\color{black}\{Fu\ss notentext\}\end{ttfamily} Erstellt eine Fußnote an dieser Stelle mit automatischer Nummerierung.

\begin{ttfamily}\color{nounibaredI}\textbackslash footnote\color{nounibagreenI}[Zahl]\color{black}\{Fu\ss notentext\}\end{ttfamily} Eine manuelle Nummerierung ist ebenfalls möglich.

\begin{ttfamily}\color{nounibaredI}\textbackslash footnotemark\color{nounibagreenI}[Zahl]\color{black}\end{ttfamily}
Eine Zahl kann auch ohne Fußnotentext eingetragen werden
\end{column}
\end{columns}
\bigskip
Die Nummerierung erfolgt automatisch und ist fortlaufend, unabhängig davon, ob
eine neue Seite oder {\ttfamily section} beginnt.
\end{frame}
\section{$\mathcal{A}2$} 
\begin{frame}
\frametitle{Aufgabe 2}
\framesubtitle{Baut Aufgabe2.pdf mit \LaTeX ~nach!} 

\begin{block}{\"Ubung 2}
\begin{itemize}
  \item Benennt Eure Dateien einheitlich
  \item Verwendet passende Abschnittsbefehle
  \item Wenn was schief l\"auft, schaut in der Konsole nach
  \item \"Ubung macht den Meister!
\end{itemize}
\end{block}
\begin{alertblock}{Der Beginn der \LaTeX -Datei sollte immer folgendes enthalten:}
\begin{ttfamily}\color{nounibaredII}\textbackslash documentclass\color{nounibagreenI}[a4paper, pdftex, 12pt, ngerman]\color{black}\{article\}\\
$\color{nounibaredII}\backslash$\color{nounibaredII}usepackage\color{nounibagreenI}[utf8]\color{black}\{inputenc\}\\
$\color{nounibaredII}\backslash$\color{nounibaredII}usepackage\color{nounibagreenI}[T1]\color{black}\{fontenc\}\\
$\color{nounibaredII}\backslash$\color{nounibaredII}usepackage\color{black}\{babel\}\\\end{ttfamily}
\end{alertblock}
\end{frame}
\section{Grafiken \& Referenzen}
\begin{frame}
\frametitle{Grafiken}
\framesubtitle{Einbinden von Grafiken} 
\begin{exampleblock}{Neue Pakete in diesem Abschnitt}
\begin{itemize}
\item graphicx 
\item float
\end{itemize}
\end{exampleblock}

\begin{block}{Neue Befehle in diesem Abschnitt}
\begin{itemize}
\item \color{nounibaredI}\textbackslash includegraphics\color{black}\{Datei\}
\item \color{nounibaredI}\textbackslash caption\color{black}\{Bildunterschrift\}
\item \begin{ttfamily}\color{nounibaredI}\textbackslash label\color{black}\{Label\}
\item \color{nounibaredI}\textbackslash ref\color{black}\{Referenz\}\end{ttfamily}
\end{itemize}
\end{block}

\end{frame}

%-------------------------------------------------------------------------------

\begin{frame}
\frametitle{Grafiken}
\framesubtitle{Abbildungen einfügen}
\begin{tabbing}
\textbackslash includegraphics[option]\{Datei\}xx\=\=\=\kill
\begin{ttfamily}
\color{unibablueI}\textbackslash begin\color{black}\{figure\}\color{nounibagreenI}[option]\color{black}
\end{ttfamily}
\>\>\textbf{Mögliche Optionen für die Positionierung:}\\
\>\>\begin{ttfamily}\color{nounibagreenI}[h]\color{black}\end{ttfamily} = Hier an dieser Stelle\\
\>\>\begin{ttfamily}\color{nounibagreenI}[t]\color{black}\end{ttfamily} = Oben auf der Seite\\
\>\>\begin{ttfamily}\color{nounibagreenI}[b]\color{black}\end{ttfamily} = Unten auf der Seite\\
\>\>\begin{ttfamily}\color{nounibagreenI}[p]\color{black}\end{ttfamily} = Platzierung auf einer eigenen
Seite\\[5mm]
~\\[5mm]
\color{nounibaredI}\textbackslash includegraphics\color{nounibagreenI}[option]\color{black}\{datei\}
\>\>\textbf{Mögliche Optionen für das Einfügen:}\\
\>\>\begin{ttfamily}\color{nounibagreenI}[width=300pt]\color{black}\end{ttfamily}= Skalieren auf eine Breite\\
\>\>\begin{ttfamily}\color{nounibagreenI}[height=5cm]\color{black}\end{ttfamily}= Skalieren auf eine Höhe\\
\>\>scale, angle und noch viele mehr\ldots\\
\>\>Kombinationen möglich:\\
\>\>\begin{ttfamily}\color{nounibagreenI}[width=\textbackslash textwidth,height=5cm]\color{black}\end{ttfamily}
\end{tabbing}
\end{frame}

%-------------------------------------------------------------------------------

\begin{frame}
\frametitle{Grafiken}
\framesubtitle{Positionierung von Abbildungen}
\begin{columns}
\begin{column}{.5\textwidth}
{\ttfamily {\footnotesize
\color{nounibaredI}\color{nounibaredI}\textbackslash documentclass\color{black}\{article\} \\
\color{nounibaredI}\color{nounibaredI}\textbackslash usepackage\color{black}\{graphicx\} \\
\color{nounibaredI}\color{unibablueI}\textbackslash\color{unibablueI}begin\color{black}\color{black}\{document\} \\
\color{nounibaredI}\color{unibablueI}\textbackslash\color{unibablueI}begin\color{black}\color{black}\{figure\}\color{nounibagreenI}[h]\color{black} \\
\color{nounibaredI}\color{unibablueI}\textbackslash\color{unibablueI}begin\color{black}\color{black}\{center\} \\
	\color{nounibaredI}\color{nounibaredI}\textbackslash includegraphics\color{black}\color{nounibagreenI}[width=50mm]\color{black}\{tux.png\} \\
\color{nounibaredI}\color{nounibaredI}\textbackslash caption\color{black}\{Tiny Tux\} \\
\color{nounibaredI}\color{nounibaredI}\textbackslash label\color{black}\{img:tinytux\} \\
\color{nounibaredI}\color{unibablueI}\textbackslash\color{unibablueI}end\color{black}\color{black}\{center\} \\
\color{nounibaredI}\color{unibablueI}\textbackslash\color{unibablueI}end\color{black}\color{black}\{figure\} \\
\color{nounibaredI}\color{unibablueI}\textbackslash\color{unibablueI}end\color{black}\color{black}\{document\} \\
}}

\begin{alertblock}{Obacht!}
\color{nounibaredI}\textbackslash label\color{black}\{\} immer nach \color{nounibaredI}\textbackslash caption\color{black}\{\}
\end{alertblock}
\end{column}

\begin{column}{.5\textwidth} 
\begin{figure}
\begin{center}
    \includegraphics[width=\textwidth]{image/tux.png}
\caption{Der kleine Tux}
\label{img:kleinertux}
\end{center}
\end{figure}
\end{column}
\end{columns}
\end{frame}

%-------------------------------------------------------------------------------

\begin{frame}[t]
\medskip
\frametitle{Grafiken}
\framesubtitle{Positionierung von Abbildungen II }
Trotz Definition einer Positionsumgebung verrutscht das Bild oft, da es nicht immer m\"oglich ist das Bild an passender Stelle einzufügen.\\
\textbf{L\"osung:} Das Packet {\ttfamily float} liefert in den meisten F\"allen bessere Positionierungen.

\begin{columns}
\begin{column}{.5\textwidth}
{\ttfamily {\footnotesize
\color{nounibaredI}\color{nounibaredI}\textbackslash documentclass\color{black}\{article\} \\
\color{nounibaredI}\color{nounibaredI}\textbackslash usepackage\color{black}\{graphicx\} \\
\color{nounibaredI}\color{nounibaredI}\textbackslash usepackage\color{black}\{float\} \\
\color{nounibaredI}\color{unibablueI}\textbackslash\color{unibablueI}begin\color{black}\color{black}\{document\} \\
\color{nounibaredI}\color{unibablueI}\textbackslash\color{unibablueI}begin\color{black}\color{black}\{figure\}\color{nounibagreenI}[H]\color{black} \\
\color{nounibaredI}\color{unibablueI}\textbackslash\color{unibablueI}begin\color{black}\color{black}\{center\} \\
	\color{nounibaredI}\color{nounibaredI}\textbackslash includegraphics\color{black}\color{nounibagreenI}[width=70mm]\color{black}\{pfad/tux.png\} \\
\color{nounibaredI}\color{nounibaredI}\textbackslash caption\color{black}\{Der kleine Tux jetzt in Float\} \\
\color{nounibaredI}\color{nounibaredI}\textbackslash label\color{black}\{img:kleinertux-float\} \\
\color{nounibaredI}\color{unibablueI}\textbackslash\color{unibablueI}end\color{black}\color{black}\{center\} \\
\color{nounibaredI}\color{unibablueI}\textbackslash\color{unibablueI}end\color{black}\color{black}\{figure\} \\
\color{nounibaredI}\color{unibablueI}\textbackslash\color{unibablueI}end\color{black}\color{black}\{document\} \\
}}
\end{column}

\begin{column}{.5\textwidth} 
\begin{figure}
\begin{center}
    \includegraphics[width=35mm]{image/tux.png}
\caption{Der kleine Tux jetzt in Float}
\label{img:kleinertux_float}
\end{center}
\end{figure}
\end{column}
\end{columns}

\end{frame}


\begin{frame}
\frametitle{Referenzen}
\framesubtitle{Abbildungen einfügen – A closer look}
\begin{tabbing}
\begin{ttfamily}\color{nounibaredI}\textbackslash label\color{black}\{Labelname\}\end{ttfamily} \=Mit diesem Befehl
setzt man ein Label. Später im\\\> Text kann man dann durch eine Referenz auf dieses\\
\> Label verweisen.\\
\> Dies geschieht mit dem Befehl
\begin{ttfamily}\color{nounibaredI}\textbackslash ref\color{black}\{Labelname\}\end{ttfamily}.
\end{tabbing}
\begin{ttfamily}Der kleine Tux ist ein Allesfresser. Egal ob Gem"use oder
Schnittlauch, nichts ist vor ihm sicher. (siehe Bild
\color{nounibaredI}\textbackslash ref\color{black}\{img:tux1\})\end{ttfamily}\\[3mm]
\textbf{Ergebnis:}\\[3mm]
\begin{minipage}{\textwidth}\begin{rm}
Der kleine Tux ist ein Allesfresser. Egal ob Gem"use oder
Schnittlauch, nichts ist vor ihm sicher. (siehe Bild
1)\end{rm} \end{minipage}\\[3mm]
%\begin{exampleblock}{Und warum das Ganze?}
%Durch solche Referenzen wird immer auf das richtige Bild verwiesen, auch wenn zwischendurch noch weitere Bilder einfügt wurden.
%\end{exampleblock}
\textbf{Und warum das Ganze?}\\
Durch solche Referenzen wird immer auf das richtige Bild verwiesen, auch wenn zwischendurch noch weitere Bilder einfügt wurden.
\end{frame}

%-------------------------------------

%\begin{frame}
%\frametitle{Referenzen}
%\framesubtitle{Packet cRef}
%
%Mit dem Package cleveref wird die Referenz direkt in der richtigen Sprache beschriftet.\\[3mm]
%
%\begin{ttfamily}siehe 
%\color{nounibaredI}\textbackslash cref\color{black}\{img:tux1\}\end{ttfamily}\\[3mm]
%\textbf{Ergebnis:}\\[3mm]
%\begin{minipage}{\textwidth}\begin{rm}
%siehe \cref{img:kleinertux}\end{rm} \end{minipage}\\[3mm]
%
%\end{frame}

\section{$\mathcal{A}3$} 
\begin{frame}
\frametitle{Aufgabe 3}
\framesubtitle{Baut Aufgabe3.pdf in \LaTeX ~nach!} 

\begin{block}{Aufgabe 3}
\begin{itemize}
\item Versucht die Autovervollst"andigungsfunktion des \TeX maker zu lernen und einzusetzen
\item Nicht vergessen die neuen Packages einzubinden
\item Achtet auf eine intuitive Benennung der Labels
\end{itemize}
\end{block}
\end{frame}
\section{Formeln}
\begin{frame}
\frametitle{Formeln}
\framesubtitle{Mathematische Formeln einbinden}

\begin{exampleblock}{Neue Pakete in diesem Abschnitt}
\begin{multicols}{2}
\begin{itemize}
\item amsmath 
\item amsthm
\item amssymb
\item mathtools
\end{itemize}
\end{multicols}
\end{exampleblock}

\begin{block}{Neue Befehle in diesem Abschnitt}
\begin{multicols}{2}
\begin{itemize}
\item \color{nounibaredI}\textbackslash sqrt\color{black}\{\}
\item \color{nounibaredI}\textbackslash frac\color{black}\{\}\{\}
\item \color{nounibaredI}\textbackslash int\color{black}\_X
\item \color{nounibaredI}\textbackslash sum\color{black}\_\{\}
\item \color{nounibaredI}\textbackslash lim\color{black}\_\{\}
\item \color{nounibaredI}\textbackslash prod\color{black}
\item \color{nounibaredI}\textbackslash limits\color{black}\_\{\}
\item \color{nounibaredI}\textbackslash dots\color{black}
\item \color{nounibaredI}\textbackslash cdot\color{black}
\item \color{nounibaredI}\_\color{black}
\item \color{nounibaredI}\^~\color{black}
\end{itemize}
\end{multicols}
\end{block}

\end{frame}

%-------------------------------------------------------------------------------
\begin{frame}
\frametitle{Formeln}
\framesubtitle{\ldots ~in \LaTeX ~eine wahre Sch\"onheit!}

\begin{columns}
\begin{column}{.3\textwidth}
{\huge $2 \sqrt{\frac{\pi ^2}{3}\cdot c_{2}}$}
\end{column}

\begin{column}{.7\textwidth}
$\underbrace{\color{unibayellowI}\text{\$}%
\color{black}2%
\color{nounibaredI}\backslash \text{sqrt}%
\color{black}\{%
\color{nounibaredI}\backslash \text{frac}%
\color{black}\{%
\color{nounibaredI}\backslash \text{pi}\color{nounibaredI}\^~{}\color{black}2\}\{3\color{black}\}%
\color{nounibaredI}\backslash%
\color{nounibaredI}\text{cdot}~%
\color{black} c%\color{nounibaredI}\_%
\color{black}2\}%
\color{unibayellowI}\$%
}$\color{black}

Die Formel-Umgebung wird durch \color{unibayellowI}\$ \color{black} angefangen und beendet.

\medskip
$\color{unibayellowI}\$\color{black}\underbrace{2\color{nounibaredI}$\color{nounibaredI}\backslash$sqrt\color{black}\{\color{nounibaredI}$\color{nounibaredI}\backslash$frac\color{black}\{\color{nounibaredI}$\color{nounibaredI}\backslash$pi\color{nounibaredI}\^ {}\color{black}2\}\{3\color{black}\}\color{nounibaredI}$\color{nounibaredI}\backslash$cdot~\color{black} c\color{nounibaredI}\_\color{black}2\}$}\color{unibayellowI}\$\color{black} 

So weit reicht die Wurzel.

\bigskip
$\color{unibayellowI}\$\color{black}2\color{nounibaredI}$\color{nounibaredI}\backslash$sqrt\color{black}\underbrace{\{\color{nounibaredI}$\color{nounibaredI}\backslash$frac\color{black}\{\color{nounibaredI}$\color{nounibaredI}\backslash$pi\color{nounibaredI}\^ {}\color{black}2\}\{3\color{black}\}\color{nounibaredI}$}\color{nounibaredI}\backslash$\color{nounibaredI}cdot~\color{black} c\color{nounibaredI}\_\color{black}2\}$\color{unibayellowI}\$\color{black} 

Ein Bruch hat immer Z"ahler und Nenner.
\end{column}
\end{columns}
\end{frame}

%-------------------------------------------------------------------------------

\begin{frame}
\frametitle{Formeln}
\framesubtitle{\ldots ~in \LaTeX ~eine wahre Sch\"onheit!}
\begin{columns}
\begin{column}{.4\textwidth}
\flushright
$\int_0^\infty$
\end{column}
\begin{column}{.6\textwidth}
\flushleft
{\ttfamily\color{unibayellowI}\$\color{nounibaredI}\textbackslash\color{nounibaredI}int\_\color{black}0\color{nounibaredI}\textasciicircum \textbackslash infty\color{unibayellowI}\$}
\end{column}
\end{columns}
\begin{columns}
\begin{column}{.4\textwidth}
\flushright
$\sum_{i=1}^n$
\end{column}
\begin{column}{.6\textwidth}
\flushleft
{\ttfamily $\color{unibayellowI}\$$\color{nounibaredI}\backslash$\color{nounibaredI}sum\_\color{black}\{i=1\}\color{nounibaredI}\^{}\color{black}n\color{unibayellowI}\$}
\end{column}
\end{columns}

\begin{columns}
\begin{column}{.4\textwidth}
\flushright
$\lim_{n \rightarrow \infty}$
\end{column}
\begin{column}{.6\textwidth}
\flushleft
{\ttfamily $\color{unibayellowI}\$$\color{nounibaredI}\backslash$\color{nounibaredI}lim\_\color{black}\{n $\color{nounibaredI}\backslash$\color{nounibaredI}rightarrow $\color{nounibaredI}\backslash$infty\color{black}\}\color{unibayellowI}\$}
\end{column}
\end{columns}

\begin{columns}
\begin{column}{.4\textwidth}
\flushright
$\prod\limits_{i=1}^{n+1}i = 1 \cdot 2 \cdot \ldots \cdot n \cdot (n+1)$
\end{column}
\begin{column}{.6\textwidth}
\flushleft
{\ttfamily $\color{unibayellowI}\$$\color{nounibaredI}\backslash$\color{nounibaredI}prod\textbackslash limits\_\color{black}\{i=1\}\color{nounibaredI}\^{}\color{black}\{n+1\}i = 1 \color{nounibaredI}\backslash$\color{nounibaredI}cdot \color{black}2 \color{nounibaredI}\backslash$\color{nounibaredI}cdot \color{nounibaredI}\backslash$\color{nounibaredI}ldots \color{nounibaredI}\backslash$\color{nounibaredI}cdot \color{black}n \color{nounibaredI}\backslash$\color{nounibaredI}cdot \color{black}(n+1)\color{unibayellowI}\$}
\end{column}
\end{columns}
\bigskip
Die  American Mathematical Society hat einen wundersch"onen Guide f"ur das {\ttfamily amsmath}-Package.\footnote{ftp://ftp.ams.org/pub/tex/doc/amsmath/amsldoc.pdf}
\end{frame}
\section{Code}

\begin{frame}
\frametitle{Code}
\framesubtitle{Programmcode darstellen}

\begin{exampleblock}{Neue Pakete in diesem Abschnitt}
\begin{multicols}{3}
\begin{itemize}
\item verbatim
\item listings
\item color
\end{itemize}
\end{multicols}
\end{exampleblock}

\begin{block}{Neue Befehle in diesem Abschnitt}
%\begin{multicols}{2}
\begin{itemize}
\item \color{unibablueI}\textbackslash begin\color{black}\{verbatim\} \dots
~\color{unibablueI}\textbackslash end\color{black}\{verbatim\}
\item \color{unibablueI}\textbackslash begin\color{black}\{lstlisting\} \dots
~\color{unibablueI}\textbackslash end\color{black}\{lstlisting\}
\item \color{nounibaredI}\textbackslash color\color{black}\{\}
\item \color{nounibaredI}\textbackslash lstset\color{black}\{\}
%TODO mehr?
\end{itemize}
%\end{multicols}
\end{block}

\end{frame}

%-----------------------------------------------------------------------------

\begin{frame}[fragile]
\frametitle{Code}
\framesubtitle{Unformatierte Texte \& Codeabschnitte}
Die Verbatim Umgebung:\\
\begin{columns}
\begin{column}{.5\textwidth}
\begin{verbatim}
Dieser Satz erscheint
so im Text.
Auch Befehle wie
\textbf{werden}
nicht interpretiert.
\end{verbatim}
\end{column}
\begin{column}{.5\textwidth}
\begin{ttfamily}
\begin{tabbing}
x\=\kill\\
\>\color{unibablueI}\textbackslash begin\color{black}\{verbatim\}\\
\>Dieser Satz erscheint\\
\>so im Text.\\
\>Auch Befehle wie\\
\>\textbackslash textbf\{werden\}\\
\>nicht interpretiert.\\
\>\color{unibablueI}\textbackslash end\color{black}\{verbatim\}\\
\end{tabbing}
\end{ttfamily}
\end{column}
\end{columns}
\end{frame}

%-------------------------------------------------------------------------------

\begin{frame}[fragile]
\frametitle{Code}
\framesubtitle{Unformatierte Texte \& Codeabschnitte}
\vspace{3mm}
%Die Verbatim Umgebung (Paket: verbatim):\\
\scriptsize
\lstset{language=Java, commentstyle=\color{green}}
\begin{lstlisting}
public static void printNumber(int n)
{
    for (int i = 0; i < n; i++)
    {
        // print the current number
        System.out.println("Number: " + i);
    }
}
\end{lstlisting}

\footnotesize
\vspace{-2mm}

\begin{ttfamily}
\begin{tabbing}
xx\=xx\=\kill\\
\color{nounibaredI}\textbackslash usepackage\color{black}\{listings\}\\
\color{nounibaredI}\textbackslash usepackage\color{black}\{color\}\\
\color{nounibaredI}\textbackslash lstset\color{black}\{language=Java, commentstyle=\color{nounibaredI}\textbackslash color\color{black}\{green\}\}\\
\color{unibablueI}\textbackslash begin\color{black}\{lstlisting\}\\
public static void printNumber(int n)\\
\{\\
\>for (int i = 0; i < n; i++)\\
\>\{\\
\>\>// print the current number\\
\>\>System.out.println(\verb|"|Number: \verb|"| + i);\\
\>\}\\
\}\\
\color{unibablueI}\textbackslash end\color{black}\{lstlisting\}\\
\end{tabbing}
\end{ttfamily}
\normalsize
\end{frame}



%TODO hinweis auf weitere Formatierungsmöglichkeiten -> syntax hilighting (siehe Internet)


\section{Tables}

\begin{frame}
\frametitle{Tables}
\framesubtitle{Insertion Of Tables}

\begin{exampleblock}{New packages in this section}
\begin{itemize}
\item longtable
\end{itemize}
\end{exampleblock}

\begin{block}{New commands in this section}
\begin{itemize}
\item \color{unibablueI}\textbackslash begin\color{black}\{tabular\} \dots
~\color{unibablueI}\textbackslash end\color{black}\{tabular\}
\item \color{unibablueI}\textbackslash begin\color{black}\{table\} \dots
~\color{unibablueI}\textbackslash end\color{black}\{table\}
\item \color{unibablueI}\textbackslash begin\color{black}\{longtable\} \dots
~\color{unibablueI}\textbackslash end\color{black}\{longtable\}
\item \color{unibablueI}\textbackslash begin\color{black}\{tabbing\} \dots
~\color{unibablueI}\textbackslash end\color{black}\{tabbing\}
\item \color{nounibaredI}$|$\color{black}
\item \color{nounibaredI}\& \color{black}
\item \color{nounibaredI}\textbackslash hline\color{black}
\item \color{nounibaredI}\textbackslash multicolumn\color{black}\{\}\{\}\{\}
\end{itemize}
\end{block}

\end{frame}

\begin{frame}
\frametitle{Tables}
\framesubtitle{``table'' \& \glqq tabular\grqq}
\textbf{Structure:}\\[2mm]
\color{unibablueI}\begin{ttfamily}\textbackslash begin\color{black}\{table\}\color{nounibagreenI}[position]\color{black}\\
\color{unibablueI}\textbackslash begin\color{black}\{tabular\}\{\textit{Definition of columns}\}\\
\textit{Table content}\\
\color{unibablueI}\textbackslash end\color{black}\{tabular\}\\
\color{nounibaredI}\textbackslash caption\color{black}\{caption\}\\
\color{nounibaredI}\textbackslash label\color{black}\{tab:bsptab1\}\\
\color{unibablueI}\textbackslash end\color{black}\{table\}\\
~\\
\end{ttfamily}

\begin{block}{Reminder: positioning in most \LaTeX -- environments}
\color{nounibagreenI}[h]\color{black}~or \color{nounibagreenI}[H]\color{black}~= At this very position\\
\color{nounibagreenI}[t]\color{black}~= On top of the page\\ 
\color{nounibagreenI}[b]\color{black}~= On bottom of the page\\ 
\color{nounibagreenI}[p]\color{black}~= Positioning on an own page
\end{block}
\end{frame}


\begin{frame}
\frametitle{Tables}
\framesubtitle{Definition Of Columns}
Here you can define how the columns should be aligned
and how the vertical lines should be set:\\[3mm]
\begin{tabbing}[H]p{column width}xxx\=\kill
\textbf{Commands:}\\
l \>= left-justified\\
c \>= centered\\
r \>= right-justified\\
p\{column width\} \>= a left-justified column with defined width\\
\color{nounibaredI}$|$\color{black} \>= sets a vertical line
at this position\\
\end{tabbing}
\end{frame}

\begin{frame}
\frametitle{Tables}
\framesubtitle{View From Inside}
\begin{ttfamily}
\color{nounibaredI}\color{unibablueI}\textbackslash\color{unibablueI}begin\color{black}\color{black}\{tabular\}\{c\color{nounibaredI}|\color{black}p\{40mm\}\color{nounibaredI}|\color{black}lr\color{nounibaredI}|\color{black}c\} \\
\color{nounibaredI}\color{nounibaredI}\textbackslash multicolumn\color{black}\{5\}\{c\}\{E-Sports Championship Franconia\} \color{nounibaredI}\color{nounibaredI}\textbackslash \color{nounibaredI}\textbackslash \color{black} \\
\color{nounibaredI}\color{nounibaredI}\textbackslash hline\color{black} \\
\color{nounibaredI}\color{nounibaredI}\textbackslash hline\color{black} \\
Number \color{nounibaredI}\&  \color{black}Place \color{nounibaredI}\&  \color{black}Player 1 \color{nounibaredI}\&  \color{black}Player 2 \color{nounibaredI}\&  \color{black}Result \color{nounibaredI}\color{nounibaredI}\textbackslash \color{nounibaredI}\textbackslash \color{black} \\
\color{nounibaredI}\color{nounibaredI}\textbackslash hline\color{black} \\
1 \color{nounibaredI}\&  \color{black}Nürnberg \color{nounibaredI}\&  \color{black}Wolf \color{nounibaredI}\&  \color{black}Lamm \color{nounibaredI}\&  \color{black}23:10 \color{nounibaredI}\color{nounibaredI}\textbackslash \color{nounibaredI}\textbackslash \color{black} \\
\color{nounibaredI}\color{nounibaredI}\textbackslash hline\color{black} \\
2 \color{nounibaredI}\&  \color{black}Bamberg \color{nounibaredI}\&  \color{black}Meyer \color{nounibaredI}\&  \color{black}Beyer \color{nounibaredI}\color{nounibaredI}\textbackslash \color{nounibaredI}\textbackslash \color{black} \\
\color{nounibaredI}\color{nounibaredI}\textbackslash hline\color{black} \\
3 \color{nounibaredI}\&  \color{black}Zirndorf \color{nounibaredI}\&  \color{black}Brandst. \color{nounibaredI}\&  \color{black}Brauer \color{nounibaredI}\&  \color{black}21:21\color{nounibaredI}\color{nounibaredI}\textbackslash \color{nounibaredI}\textbackslash \color{black} \\
\color{nounibaredI}\color{nounibaredI}\textbackslash hline\color{black} \\
\color{nounibaredI}\color{unibablueI}\textbackslash\color{unibablueI}end\color{black}\color{black}\{tabular\} \\

\end{ttfamily}
\end{frame}

\begin{frame}
\frametitle{Tables}
\framesubtitle{Table Content}
Here you fill the defined columns with content.\\[3mm]
\begin{tabbing}[H]p{column width}xxx\=\kill
\textbf{Commands:}\\
\color{nounibaredI}\&\color{black} \>= horizontal separation of rows\\
\color{nounibaredI}\textbackslash \textbackslash \color{black} \>=  new line\\
\color{nounibaredI}\textbackslash hline\color{black} \>= sets a horizontal line\\[2mm]
\color{nounibaredI}\textbackslash multicolumn\color{black}\{column number\}\{column alignment\}\{text\}\\[2mm]
\>= Combines as many columns as you like.\\
\end{tabbing}
\end{frame}

\begin{frame}[t]

\frametitle{Tables}
\framesubtitle{Example Tabular}
\begin{footnotesize}
\begin{ttfamily}
\color{nounibaredI}\color{unibablueI}\textbackslash\color{unibablueI}begin\color{black}\color{black}\{tabular\}\{c\color{nounibaredI}|\color{black}p\{40mm\}\color{nounibaredI}|\color{black}lr\color{nounibaredI}|\color{black}c\} \\
\color{nounibaredI}\color{nounibaredI}\textbackslash multicolumn\color{black}\{5\}\{c\}\{E-Sports Championship Franconia\} \color{nounibaredI}\color{nounibaredI}\textbackslash \color{nounibaredI}\textbackslash \color{black} \\
\color{nounibaredI}\color{nounibaredI}\textbackslash hline\color{black} \\
\color{nounibaredI}\color{nounibaredI}\textbackslash hline\color{black} \\
Number \color{nounibaredI}\&  \color{black}Place \color{nounibaredI}\&  \color{black}Player 1 \color{nounibaredI}\&  \color{black}Player 2 \color{nounibaredI}\&  \color{black}Result \color{nounibaredI}\color{nounibaredI}\textbackslash \color{nounibaredI}\textbackslash \color{black} \\
\color{nounibaredI}\color{nounibaredI}\textbackslash hline\color{black} \\
1 \color{nounibaredI}\&  \color{black}Nürnberg \color{nounibaredI}\&  \color{black}Wolf \color{nounibaredI}\&  \color{black}Lamm \color{nounibaredI}\&  \color{black}23:10 \color{nounibaredI}\color{nounibaredI}\textbackslash \color{nounibaredI}\textbackslash \color{black} \\
\color{nounibaredI}\color{nounibaredI}\textbackslash hline\color{black} \\
2 \color{nounibaredI}\&  \color{black}Bamberg \color{nounibaredI}\&  \color{black}Meyer \color{nounibaredI}\&  \color{black}Beyer \color{nounibaredI}\color{nounibaredI}\textbackslash \color{nounibaredI}\textbackslash \color{black} \\
\color{nounibaredI}\color{nounibaredI}\textbackslash hline\color{black} \\
3 \color{nounibaredI}\&  \color{black}Zirndorf \color{nounibaredI}\&  \color{black}Brandst. \color{nounibaredI}\&  \color{black}Brauer \color{nounibaredI}\&  \color{black}21:21\color{nounibaredI}\color{nounibaredI}\textbackslash \color{nounibaredI}\textbackslash \color{black} \\
\color{nounibaredI}\color{nounibaredI}\textbackslash hline\color{black} \\
\color{nounibaredI}\color{unibablueI}\textbackslash\color{unibablueI}end\color{black}\color{black}\{tabular\} \\

\end{ttfamily}
\end{footnotesize}

\begin{tabular}{c|p{40mm}|lr|c}
\multicolumn{5}{c}{E-Sports Championship Franconia}
 \\
\hline
\hline
Number & Place & Player 1 & Player 2 & Result \\
\hline
1 & N\"urnberg & Wolf & Lamm & 23:10 \\
\hline
2 & Bamberg & Meyer & Beyer & \\
\hline
3 & Zirndorf & Brandst. & Brauer & 21:21 \\
\hline
\end{tabular}
\end{frame}

\begin{frame}{Tables}
\framesubtitle{Longtable -- Table With Line Break}
\bigskip
„tabular“ shows the table on one page. If it does not fit on the page, the remainder is cut off.\\
For tables longer than one page, a table is needed that performs a division of the table.\\
\textbf{Solution: {\ttfamily longtable}}\\
{\ttfamily longtable} allows a line break in the table. Moreover, {\ttfamily longtable} is an environment, so the  {\ttfamily table}-environment is not needed anymore!\\[3mm]

\begin{ttfamily}
\color{unibablueI}\textbackslash begin\color{black}\{longtable\}\{\textit{definition of columns}\}\\
\textit{table content}\\
\color{nounibaredI}\textbackslash caption\color{black}\{caption\}\\
\color{nounibaredI}\textbackslash label\color{black}\{tab:bsptab2\}\\
\color{unibablueI}\textbackslash end\color{black}\{longtable\}
\end{ttfamily}
\end{frame}


\begin{frame}
\frametitle{Indentations With „tabbing“}
\begin{block}{Control}
\begin{itemize}
\item[\begin{ttfamily}\color{nounibaredI}\textbackslash =\end{ttfamily}]\color{black}set a tab position 
\item[\begin{ttfamily}\color{nounibaredI}\textbackslash $>$\end{ttfamily}]select a tab position
\end{itemize}
\end{block}

\begin{columns}
\begin{column}{.5\textwidth}
\begin{ttfamily}{\scriptsize
\color{nounibaredI}\color{nounibaredI}\textbackslash documentclass\color{black}\{article\} \\
\color{nounibaredI}\color{unibablueI}\textbackslash\color{unibablueI}begin\color{black}\color{black}\{document\} \\
\color{nounibaredI}\color{unibablueI}\textbackslash\color{unibablueI}begin\color{black}\color{black}\{tabbing\} \\
Mitarb\color{nounibaredI}\color{nounibaredI}\textbackslash \color{black}=eiter:\color{nounibaredI}\color{nounibaredI}\textbackslash \color{nounibaredI}\textbackslash \color{black} \\
A  \color{nounibaredI}\color{nounibaredI}\textbackslash \color{black}> Daniel\color{nounibaredI}\color{nounibaredI}\textbackslash \color{nounibaredI}\textbackslash \color{black} \\
B  \color{nounibaredI}\color{nounibaredI}\textbackslash \color{black}> Martin\color{nounibaredI}\color{nounibaredI}\textbackslash \color{nounibaredI}\textbackslash \color{black} \\
C  \color{nounibaredI}\color{nounibaredI}\textbackslash \color{black}> Linus\color{nounibaredI}\color{nounibaredI}\textbackslash \color{nounibaredI}\textbackslash \color{black} \\
xxx\color{nounibaredI}\color{nounibaredI}\textbackslash \color{black}=xxx\color{nounibaredI}\color{nounibaredI}\textbackslash \color{black}=xxxxxxx\color{nounibaredI}\color{nounibaredI}\textbackslash kill\color{black} \\
\color{nounibaredI}\color{nounibaredI}\textbackslash \color{black}> Gremien\color{nounibaredI}\color{nounibaredI}\textbackslash \color{nounibaredI}\textbackslash \color{black} \\
\color{nounibaredI}\color{nounibaredI}\textbackslash \color{black}>\color{nounibaredI}\color{nounibaredI}\textbackslash \color{black}> Klausuren\color{nounibaredI}\color{nounibaredI}\textbackslash \color{nounibaredI}\textbackslash \color{black} \\
\color{nounibaredI}\color{nounibaredI}\textbackslash \color{black}>\color{nounibaredI}\color{nounibaredI}\textbackslash \color{black}> Emails\color{nounibaredI}\color{nounibaredI}\textbackslash \color{nounibaredI}\textbackslash \color{black} \\
\color{nounibaredI}\color{unibablueI}\textbackslash\color{unibablueI}end\color{black}\color{black}\{tabbing\} \\
\color{nounibaredI}\color{unibablueI}\textbackslash\color{unibablueI}end\color{black}\color{black}\{document\} \\
}
\end{ttfamily}
\end{column}
\begin{column}{.5\textwidth}
\begin{tabbing}
Employ\=ee:\\
A  \> Daniel\\
B  \> Martin\\
C  \> Linus\\
xxx\=xxx\=xxxxxxx\kill
\> Committees\\
\>\> Tests\\
\>\> Mails\\
\end{tabbing}
\end{column}
\end{columns}

If you use the command \begin{ttfamily}\color{nounibaredI}\textbackslash kill\color{black}\end{ttfamily}, the rest of the line is not shown. In this way you can do the formatting without showing the respective text.
\end{frame}

\section{Aufz\"ahlungen}
\begin{frame}
\frametitle{Aufzählungen}

\begin{block}{Neue Befehle in diesem Abschnitt}
\begin{itemize}
\begin{ttfamily}
\item \color{unibablueI}\textbackslash begin\color{black}\{itemize\} \ldots \color{unibablueI}\textbackslash end\color{black}\{itemize\} 
\item \color{unibablueI}\textbackslash begin\color{black}\{enumerate\} \ldots \color{unibablueI}\textbackslash end\color{black}\{enumerate\} 
\item \color{nounibaredI}\textbackslash item\color{black}%
\end{ttfamily}
\end{itemize}
\end{block}
\end{frame}

\begin{frame}
\frametitle{Aufzählungen}
\framesubtitle{Spiegelstrichlisten}

\begin{columns}
\begin{column}{.5\textwidth}
\begin{ttfamily}%
\color{nounibaredI}\color{nounibaredI}\textbackslash documentclass\color{black}\{article\} \\
\color{nounibaredI}\color{unibablueI}\textbackslash\color{unibablueI}begin\color{black}\color{black}\{document\} \\
\color{nounibaredI}\color{unibablueI}\textbackslash\color{unibablueI}begin\color{black}\color{black}\{itemize\} \\
\color{nounibaredI}\color{nounibaredI}\textbackslash item \color{black} erster Stichpunkt \\
\color{nounibaredI}\color{nounibaredI}\textbackslash item \color{black} zweiter Stichpunkt \\
\color{nounibaredI}\color{nounibaredI}\textbackslash item \color{black} dritter Stichpunkt \\
\color{nounibaredI}\color{nounibaredI}\textbackslash item \color{black} letzter Stichpunkt \\
\color{nounibaredI}\color{unibablueI}\textbackslash\color{unibablueI}end\color{black}\color{black}\{itemize\} \\
\color{nounibaredI}\color{unibablueI}\textbackslash\color{unibablueI}end\color{black}\color{black}\{document\} \\
\end{ttfamily}
\end{column}
\begin{column}{.5\textwidth}
\begin{itemize}
\item erster Stichpunkt
\item zweiter Stichpunkt
\item dritter Stichpunkt
\item letzter Stichpunkt
\end{itemize}
\end{column}
\end{columns}
\bigskip

Die einzelnen Stichpunkte werden innerhalb der „itemize“-Umgebung durch den Befehl \begin{ttfamily}\color{nounibaredI}\textbackslash item\color{black}\end{ttfamily} gekennzeichnet.
\end{frame}

\begin{frame}
\frametitle{Aufzählungen}
\framesubtitle{Verschachtelung}

\begin{columns}
\begin{column}{.5\textwidth}
\begin{ttfamily}%
\color{nounibaredI}\color{nounibaredI}\textbackslash documentclass\color{black}\{article\} \\
\color{nounibaredI}\color{unibablueI}\textbackslash\color{unibablueI}begin\color{black}\color{black}\{document\} \\
\color{nounibaredI}\color{unibablueI}\textbackslash\color{unibablueI}begin\color{black}\color{black}\{itemize\} \\
\color{nounibaredI}\color{nounibaredI}\textbackslash item \color{black} first bullet item \\
\color{nounibaredI}\color{nounibaredI}\textbackslash item \color{black} second bullet item \\
\color{nounibaredI}\color{unibablueI}\textbackslash\color{unibablueI}begin\color{black}\color{black}\{itemize\} \\
\color{nounibaredI}\color{nounibaredI}\textbackslash item \color{black} first subitem \\
\color{nounibaredI}\color{nounibaredI}\textbackslash item \color{black} second subitem \\
\color{nounibaredI}\color{unibablueI}\textbackslash\color{unibablueI}end\color{black}\color{black}\{itemize\} \\
\color{nounibaredI}\color{nounibaredI}\textbackslash item \color{black} third bullet item \\
\color{nounibaredI}\color{nounibaredI}\textbackslash item \color{black} last bullet item \\
\color{nounibaredI}\color{unibablueI}\textbackslash\color{unibablueI}end\color{black}\color{black}\{itemize\} \\
\color{nounibaredI}\color{unibablueI}\textbackslash\color{unibablueI}end\color{black}\color{black}\{document\} \\
\end{ttfamily}
\end{column}
\begin{column}{.5\textwidth}
\begin{itemize}
\item erster Stichpunkt
\item zweiter Stichpunkt
\begin{itemize}
\item erster Unterpunkt
\item zweiter Unterpunkt
\end{itemize}
\item dritter Stichpunkt
\item letzter Stichpunkt
\end{itemize}
\end{column}
\end{columns}
\bigskip
Auf diese Weise kann man Unterpunkte bis auf 4 Ebenen tief schachteln.
\end{frame}


\begin{frame}
\frametitle{Aufzählungen}
\framesubtitle{Nummerierungen}

\begin{columns}
\begin{column}{.5\textwidth}
\begin{ttfamily}%
\color{nounibaredI}\color{nounibaredI}\textbackslash documentclass\color{black}\{article\} \\
\color{nounibaredI}\color{unibablueI}\textbackslash\color{unibablueI}begin\color{black}\color{black}\{document\} \\
\color{nounibaredI}\color{unibablueI}\textbackslash\color{unibablueI}begin\color{black}\color{black}\{enumerate\} \\
\color{nounibaredI}\color{nounibaredI}\textbackslash item \color{black} first bullet item \\
\color{nounibaredI}\color{unibablueI}\textbackslash\color{unibablueI}begin\color{black}\color{black}\{enumerate\} \\
\color{nounibaredI}\color{nounibaredI}\textbackslash item \color{black} first subitem \\
\color{nounibaredI}\color{nounibaredI}\textbackslash item \color{black} second subitem \\
\color{nounibaredI}\color{unibablueI}\textbackslash\color{unibablueI}end\color{black}\color{black}\{enumerate\} \\
\color{nounibaredI}\color{nounibaredI}\textbackslash item \color{black} second bullet item \\
\color{nounibaredI}\color{nounibaredI}\textbackslash item \color{black} and so forth \\
\color{nounibaredI}\color{unibablueI}\textbackslash\color{unibablueI}end\color{black}\color{black}\{enumerate\} \\
\color{nounibaredI}\color{unibablueI}\textbackslash\color{unibablueI}end\color{black}\color{black}\{document\} \\
\end{ttfamily}
\end{column}
\begin{column}{.5\textwidth}
\begin{enumerate}
\item erstens
\begin{enumerate}
\item erster Unterpunkt
\item zweiter Unterpunkt
\end{enumerate}
\item zweitens
\item usw.
\end{enumerate}
\end{column}
\end{columns}
\bigskip
Auch hier werden die einzelnen Punkte durch den Befehl \begin{ttfamily}\color{nounibaredI}\textbackslash item\color{black}\end{ttfamily} gekennzeichnet. 
Schachtelungen können wieder bis zu 4 Ebenen tief sein.
\end{frame}

\begin{frame}
\frametitle{Gemischte Aufzählungen?}
\framesubtitle{Geht Alles!}

\begin{columns}
\begin{column}{.5\textwidth}
\begin{ttfamily}%
\color{nounibaredI}\color{unibablueI}\textbackslash\color{unibablueI}begin\color{black}\color{black}\{enumerate\} \\
\color{nounibaredI}\color{nounibaredI}\textbackslash item\color{black} erstens \\
\color{nounibaredI}\color{nounibaredI}\textbackslash item\color{black} \color{nounibaredI}\color{unibablueI}\textbackslash\color{unibablueI}begin\color{black}\color{black}\{itemize\} \\
\color{nounibaredI}\color{nounibaredI}\textbackslash item\color{black} erster Unterpunkt \\
\color{nounibaredI}\color{nounibaredI}\textbackslash item\color{black} zweiter Unterpunkt \\
\color{nounibaredI}\color{unibablueI}\textbackslash\color{unibablueI}end\color{black}\color{black}\{itemize\} \\
\color{nounibaredI}\color{nounibaredI}\textbackslash item\color{black} drittens \\
\color{nounibaredI}\color{unibablueI}\textbackslash\color{unibablueI}begin\color{black}\color{black}\{enumerate\} \\
\color{nounibaredI}\color{nounibaredI}\textbackslash item\color{black} Auch ich z"ahle! \\
\color{nounibaredI}\color{unibablueI}\textbackslash\color{unibablueI}end\color{black}\color{black}\{enumerate\} \\
\color{nounibaredI}\color{nounibaredI}\textbackslash item\color{black} usw. \\
\color{nounibaredI}\color{unibablueI}\textbackslash\color{unibablueI}end\color{black}\color{black}\{enumerate\} \\
\end{ttfamily}
\end{column}
\begin{column}{.5\textwidth}
\begin{enumerate}
\item erstens
\item \begin{itemize}
\item erster Unterpunkt
\item zweiter Unterpunkt
\end{itemize}
\item drittens
\begin{enumerate}
\item Auch ich z"ahle!
\end{enumerate}
\item usw.
\end{enumerate}
\end{column}
\end{columns}
\bigskip
Die Darstellung der jeweiligen Symbole kann mit \color{nounibaredI}\textbackslash item\color{nounibagreenI}[]\color{black}~angepasst werden. 
\end{frame}

\section{Aufgabe 4}
\subsection{Tabellen \& Formeln}
Die Tabelle \ref{tab:ani} besteht aus 3 Spalten:\\
Die erste Spalte ist mit einem p von 25mm definiert. Die zweite und die dritte Spalte sind zentriert.

\begin{longtable}{p{25mm}|c|c}
& Fuchs & Elster\\
\hline
\hline
Familie & Hunde & Rabenvögel\\
\hline
Gewicht & m: 6,6kg w: 5,5kg & 200--250g\\
\hline
Geschwindigkeit $ = \sqrt{v\cdot v}$ & $55\frac{km}{h}$ & mind. superschnell: $\lim\limits_{x \rightarrow \infty} x \cdot v$ \\
\hline
Farbe & tödlich & schwarz\\
\hline
\caption{Wild Animals}
\label{tab:ani}
\end{longtable}

Der Sinn des Lebens$^2$: $\prod\limits_{i=1}^{n+1} i + \sum_{j=0}^{n} j \cdot \int\limits_{\pi}^{Daumen} 42$

\subsection{Aufzählungen}
Um bei den vielen Verschachtelungen nicht den Überblick zu verlieren, sind Einrückungen der items sinnvoll.
\begin{enumerate}
  \item 
  \begin{enumerate}
    \item Vorteile des Fuchses:
    \item
    \begin{itemize}
      \item schlau
      \item schaut cool aus
    \end{itemize}
    \item Nachteile des Fuchses:
    \begin{itemize}
      \item Pelz wird verarbeitet
      \item sehr viele Autos fahren gerne über Füchse
    \end{itemize}
    \item Spam Spam Spam
  \end{enumerate}
  \item
  \begin{enumerate}
    \item Vorteile der Elster...
    \item Nachteile der Elster:
    \begin{itemize}
      \item Diebischkeit wird bestraft
      \item viele landen hinter Gittern
    \end{itemize}
      \item singt ganz gut, aber ist gefährlich
  \end{enumerate}
\end{enumerate}

\subsection{(Un)Logik}
\begin{itemize}
	\item $\lnot\forall x \Leftrightarrow \{\exists x\}$
	\item $[\exists xPx] \rightarrow \forall x \lnot Px$
\end{itemize}

\subsection{Code}
\textit{Hello World} in Java:\\
\lstset{language=Java, commentstyle=\color{green}}
\begin{lstlisting}
  public class Hello{
      public static void main(String[] args){
      
         //Hier wird der Text ausgegeben:
         System.out.println("Hello World!");
      }
  }
\end{lstlisting}
\documentclass[a4paper, pdftex, 11pt]{article}

\usepackage[utf8]{inputenc}
\usepackage[T1]{fontenc}
\usepackage[english]{babel}
\usepackage{lmodern}
\usepackage[iso]{umlaute}

%%   linker Seitenabstand 4,5cm, rechter Seitenabstand 3,5cm und Abstand zum Seitenbeginn 2,5cm   %%
\usepackage[left=45mm,right=35mm,top=25mm,bottom=25mm]{geometry}
%%   Schriftart Arial   %%
\usepackage{helvet}
\renewcommand\familydefault{phv}
%%   1,3-facher Zeilenabstand   %%
\usepackage{setspace}
\setstretch{1.3}
%%   selbstdefinierte Kopf- und Fusszeile   %%
\usepackage{fancyhdr}
\pagestyle{fancy}
\fancyhead[L]{Fachschaft WIAI}
\fancyhead[C]{University of Bamberg}
\fancyhead[R]{\today}
\fancyfoot[L]{}
\fancyfoot[C]{\thepage}
\fancyfoot[R]{}
%  Breite der Linie unter der Kopfzeile 
\renewcommand{\headrulewidth}{0pt}


%%   Einbidung der Farben und -definitionen des vordefinierten Farbschemas   %%
\usepackage{color}
\definecolor{darkred}{rgb}{.5,0,0}
\definecolor{darkgreen}{rgb}{0,.5,0}
\definecolor{darkblue}{rgb}{0,0,.5}
\usepackage[hyphens]{url}

%%   Zur Gestaltung des Textes zu einem Hypertext   %%
\usepackage{hyperref}
%\definecolor{darkblue}{rgb}{0,.05,.54}
\hypersetup{colorlinks=true, breaklinks=true, linkcolor=darkblue, menucolor=darkblue, urlcolor=darkblue, citecolor=darkblue, filecolor=darkblue}
\urlstyle{same}

\usepackage{longtable}
\usepackage{graphicx}
\usepackage{float} % e.g. documentclass, usepackage, usw. auslagern
\begin{document}
%\begin{titlepage}
\begin{center}
\Huge \LaTeX\\
\vspace{5mm} \LARGE Eine kurze Einführung\\
\vspace{12mm} \Large  Universität Bamberg\\[5mm]
\large 10. Oktober 2016\\ 
Fachschaft WIAI\normalsize \\
\end{center}
%\end{titlepage}
\tableofcontents
% Jeweils neue Seite
\listoffigures
\listoftables
\section{$\mathcal{A}1$}
\begin{frame}
\frametitle{Aufgabe 1}
\framesubtitle{Schreibt und kompiliert ``Hello World!''}
\begin{center}
\begin{rm}
\Large Hello World!\\
\end{rm}
\end{center}
\bigskip
\begin{block}{Aufgabe 1}
\begin{itemize}
\item Ladet aus dem Virtuellen Campus (\url{https://wiai.de/latex}) die Verzeichnisvorlage herunter und entpackt sie!
\item Schreibt und kompiliert \glqq{}Hello World\grqq! Ihr könnt direkt im entpackten Ordner \texttt{OrdnerStrukturVorgabe} arbeiten.
\item \textit{Hinweis:} Normale \LaTeX -Dateien haben {\ttfamily .tex} als Dateiendung
\end{itemize}
\end{block}
\begin{alertblock}{\textbf{Achtung:}}
Die Computer im PC-Pool werden automatisch zurückgesetzt. Speichert eure Daten \textbf{auf dem Heimlaufwerk \texttt{W://}}, um sie zu behalten.
\end{alertblock}
\end{frame}
\section{$\mathcal{A}2$} 
\begin{frame}
\frametitle{Aufgabe 2}
\framesubtitle{Baut Aufgabe2.pdf mit \LaTeX ~nach!} 

\begin{block}{\"Ubung 2}
\begin{itemize}
  \item Benennt Eure Dateien einheitlich
  \item Verwendet passende Abschnittsbefehle
  \item Wenn was schief l\"auft, schaut in der Konsole nach
  \item \"Ubung macht den Meister!
\end{itemize}
\end{block}
\begin{alertblock}{Der Beginn der \LaTeX -Datei sollte immer folgendes enthalten:}
\begin{ttfamily}\color{nounibaredII}\textbackslash documentclass\color{nounibagreenI}[a4paper, pdftex, 12pt, ngerman]\color{black}\{article\}\\
$\color{nounibaredII}\backslash$\color{nounibaredII}usepackage\color{nounibagreenI}[utf8]\color{black}\{inputenc\}\\
$\color{nounibaredII}\backslash$\color{nounibaredII}usepackage\color{nounibagreenI}[T1]\color{black}\{fontenc\}\\
$\color{nounibaredII}\backslash$\color{nounibaredII}usepackage\color{black}\{babel\}\\\end{ttfamily}
\end{alertblock}
\end{frame}
\section{$\mathcal{A}3$} 
\begin{frame}
\frametitle{Aufgabe 3}
\framesubtitle{Baut Aufgabe3.pdf in \LaTeX ~nach!} 

\begin{block}{Aufgabe 3}
\begin{itemize}
\item Versucht die Autovervollst"andigungsfunktion des \TeX maker zu lernen und einzusetzen
\item Nicht vergessen die neuen Packages einzubinden
\item Achtet auf eine intuitive Benennung der Labels
\end{itemize}
\end{block}
\end{frame}
% Conclusion, Literature ...
\end{document}

%% Layout 1
\documentclass[a4paper, pdftex, 11pt]{article}

\usepackage[utf8]{inputenc}
\usepackage[T1]{fontenc}
\usepackage[english]{babel}
\usepackage{lmodern}
\usepackage[iso]{umlaute}

%%   linker Seitenabstand 4,5cm, rechter Seitenabstand 3,5cm und Abstand zum Seitenbeginn 2,5cm   %%
\usepackage[left=45mm,right=35mm,top=25mm,bottom=25mm]{geometry}
%%   Schriftart Arial   %%
\usepackage{helvet}
\renewcommand\familydefault{phv}
%%   1,3-facher Zeilenabstand   %%
\usepackage{setspace}
\setstretch{1.3}
%%   selbstdefinierte Kopf- und Fusszeile   %%
\usepackage{fancyhdr}
\pagestyle{fancy}
\fancyhead[L]{Fachschaft WIAI}
\fancyhead[C]{University of Bamberg}
\fancyhead[R]{\today}
\fancyfoot[L]{}
\fancyfoot[C]{\thepage}
\fancyfoot[R]{}
%  Breite der Linie unter der Kopfzeile 
\renewcommand{\headrulewidth}{0pt}


%%   Einbidung der Farben und -definitionen des vordefinierten Farbschemas   %%
\usepackage{color}
\definecolor{darkred}{rgb}{.5,0,0}
\definecolor{darkgreen}{rgb}{0,.5,0}
\definecolor{darkblue}{rgb}{0,0,.5}
\usepackage[hyphens]{url}

%%   Zur Gestaltung des Textes zu einem Hypertext   %%
\usepackage{hyperref}
%\definecolor{darkblue}{rgb}{0,.05,.54}
\hypersetup{colorlinks=true, breaklinks=true, linkcolor=darkblue, menucolor=darkblue, urlcolor=darkblue, citecolor=darkblue, filecolor=darkblue}
\urlstyle{same}

\usepackage{longtable}
\usepackage{graphicx}
\usepackage{float}
%% Layout 2
%\documentclass[a4paper, pdftex, ngerman, 11pt]{article}
%===============================================================================
% Zweck: KTR-Seminar-Vorlage in Anlehung an G. Wirtz, Lehrstuhl Praktische Informatik
%===============================================================================
%===============================================================================
% zentrale Layout-Angaben und Befehle
%===============================================================================
%
\usepackage{babel}
\usepackage[utf8]{inputenc}
\usepackage{fancyhdr}
\usepackage[T1]{fontenc}						
\usepackage{color}
\usepackage{amsmath}
\usepackage{amsfonts}
\usepackage{float}
\usepackage{longtable}
\usepackage{multirow}							%Package zum korrekten Einfuegen von Bildern!
\usepackage[hyphens]{url}
%%   Zur Gestaltung des Textes zu einem Hypertext   %%
\usepackage{hyperref}
\usepackage{listings}
\definecolor{darkblue}{rgb}{0,.05,.54}
\hypersetup{colorlinks=true, breaklinks=true, linkcolor=darkblue, menucolor=darkblue, urlcolor=darkblue, citecolor=darkblue, filecolor=darkblue}
\urlstyle{same}

\usepackage{amsmath,amssymb,ifthen}
\usepackage{graphicx}
%
\if pdf
\usepackage[pdftex,bookmarksopen,bookmarksnumbered,pdfborder=0]{hyperref}
\pdfcompresslevel=9
\else
\usepackage{url}
\fi
%\usepackage[dvips]{rotating}
%
% ausf\"{u}hrlichere Fehlermeldungen
\errorcontextlines=999
%
% Page-Layout: A4 aus Header
% Alternative
%\setlength\headheight{14pt}
%\setlength\topmargin{-15,4mm}
%\setlength\oddsidemargin{-0,4mm}
%\setlength\evensidemargin{-0,4mm}
%\setlength\textwidth{160mm}
%\setlength\textheight{252mm}
%
% Absatzeinstellungen
\setlength\parindent{0mm}
\setlength\parskip{2ex}
%
%
% Erstellung von Abk\"{u}rzungsverzeichnis
\newcommand{\abbrev}[2]{#1 & #2\\}
\newcommand{\abkuerzungen}{
\section*{Abk\"{u}rzungsverzeichnis}
\hspace{2ex}
\begin{tabular}{ll}
\input{abkuerzungen.tex}
\end{tabular}
}
%
% Einbindung eines Bildes mit angegebener Breite
% #1 = label f\"{u}r \ref-Verweise
% #2 = Name des Bildes ohne Endung relativ zu Bilder-Verzeichnis
% #3 = Beschriftung
% #4 = Breite des Bildes im Dokument in cm
\newcommand{\bildw}[4]{%
  \begin{figure}[htb]%
    \centering
    \includegraphics[width=#4cm]{Bilder/#2}%
    \vskip -0.3cm%
    \caption{#3}%
    \vskip -0,2cm%
    \label{#1}%
  \end{figure}%
}
%
% Einbindung eines Bildes mit Seitenbreite
% #1 = label f\"{u}r \ref-Verweise
% #2 = Name des Bildes ohne Endung relativ zu Bilder-Verzeichnis
% #3 = Beschriftung
\newcommand{\bild}[3]{%
  \begin{figure}[htb]%
    \centering%
    \includegraphics[width=\textwidth]{Bilder/#2}%
    \vskip -0.3cm%
    \caption{#3}%
    \vskip -0,2cm%
    \label{#1}%
  \end{figure}%
}
%
\numberwithin{equation}{section}
%
%===============================================================================
% zentrale Layout-Angaben und Befehle
%===============================================================================
%


\begin{document}
%% Layout 1 - Titelseite
%\begin{titlepage}
\begin{center}
\Huge \LaTeX\\
\vspace{5mm} \LARGE Eine kurze Einführung\\
\vspace{12mm} \Large  Universität Bamberg\\[5mm]
\large 10. Oktober 2016\\ 
Fachschaft WIAI\normalsize \\
\end{center}
%\end{titlepage}
%% Layout 2 - Titelseite
%\begin{titlepage}
  \centering
    \begin{minipage}[t]{16cm}
      \hfill
      \begin{minipage}{12cm}
        \centering
        Otto-Friedrich-Universit\"{a}t Bamberg
        \\[12pt]
        {\Large Professur f\"ur Informatik,
        \\
        insbesondere Kommunikationsdienste, Telekommunikationssysteme
        und Rechnernetze}
      \end{minipage}
      \hfill
      \begin{minipage}{3cm}
        \includegraphics[height=28mm]{bilder/ubamlogo.png} %height=26mm
      \end{minipage}
    \end{minipage}\\[108pt]%[50pt]
   % 
    {\LARGE Ausarbeitung  im Rahmen der Veranstaltung}
    \\[36pt]
   % oder: {\LARGE Ausarbeitung des KTR-Seminars}\\[12pt]
    {\LARGE\bf Veranstaltungsname}\\[80pt]
    {\LARGE Thema:}\\[36pt]
    {\Huge Titel}\\
    \vfill
    \begin{minipage}{\textwidth}
      \center
      Vorgelegt von:\\
      {\Large Name, Vorname\\[18pt]}
      Bamberg, \today
    \end{minipage}
  \end{titlepage}


%%   Seitennummerierung mit roemischer Nummerierung   %%
\pagenumbering{Roman}
%%   Beginne die Seitennummerierung mit 2 ab dem Inhaltsverzeichnis   %%
\setcounter{page}{2}

%TODO
Hier Inhaltsverzeichnis, Abbildungsverzeichnis und Tabellenverzeichnis jeweils auf einer neuen Seite einbinden.



%%   Hauptteil   %%
%%   Seitennummerierung mit arabischer Nummerierung   %%
\pagenumbering{arabic}
%%   Beginne die Seitennummerierung mit 1 ab dem ersten eingebunden Dokument   %%
\setcounter{page}{1}

%TODO
Hier den Inhalt einbinden und für jede Datei eine neue Seite beginnen.




\textbf{Herzlichen Gl"uckwunsch, du hast das \LaTeX -Tutorium der Fachschaft WIAI bis zum Ende geschafft!}
\end{document}

\section{BibTeX}
\begin{frame}
\frametitle{BibTeX}
\framesubtitle{Add On f"ur \LaTeX}
\begin{exampleblock}{Neue Pakete in diesem Abschnitt}
\begin{itemize}
\item natbib
\end{itemize}
\end{exampleblock}

\begin{block}{Neue Befehle in diesem Abschnitt}
\begin{itemize}
\item \color{nounibaredI}\textbackslash renewcommand\color{black}\{\color{nounibaredI}\textbackslash refname\color{black}\}\{...\}
\item \color{nounibaredI}\textbackslash bibliographystyle\color{black}\{...\}
\item \color{nounibaredI}\textbackslash bibliography\color{black}\{...\}
\item \color{nounibaredI}\textbackslash addcontentsline\color{black}\{toc\}\{section\}\{\color{nounibaredI}\textbackslash refname\color{black}\}
\end{itemize}
\end{block}
\end{frame}

\begin{frame}
\frametitle{BibTeX und \LaTeX ~in Kombination}
Vorteile:
\begin{itemize}
\item Literaturverzeichnis kann in einer vom Dokument unabh"angigen Datei gespeichert werden
\item Speicherung der Daten im BibTeX-Format. Hierbei kann man nach Quellenart unterscheiden, z.B. mit $@$Book, $@$Article usw.
\item Aufnahme der Eintr"age ins Literaturverzeichnis (LVZ) des Dokuments nur dann, wenn die Quelle zitiert wurde
\item Anpassung des Aussehens des LVZ durch unterschiedlichste Style-Sheets leicht m"oglich 
\item Entwurf eigener Style-Sheets zur Gestaltung des LVZ m"oglich
\end{itemize}
Nachteile:
\begin{itemize}
\item Das Erstellen des LVZ mit den gew"unschten Anforderungen ist zum Teil nur erschwert m"oglich
\item Zum Teil "au\ss erst zeitaufw"andig
\end{itemize}
\end{frame}

\begin{frame}
\frametitle{BibTeX und \LaTeX ~in Kombination}
\framesubtitle{vgl. \url{http://de.wikipedia.org/wiki/BibTeX}}
Beispieleintrag:\\[1ex]
$@$BOOK\{\\
Culik93,\\
title = \{Die Welt der Pinguine\},\\
publisher = \{\{BLV\}M"unchen\},\\
year = \{1993\},\\
author = \{B.M. Culik and R. P. Wilson\}\}\\
\vspace*{5mm}
Erkl"arungen zum Eintrag:
\begin{itemize}
\item $@$BOOK - Angabe der Quellenart, hier also ein Buch
\item Culik93 - Definition eines eindeutigen Referenzierungsschl"ussels 
\item title - Titel des Buches
\item publisher - Verlag
\item year - Erscheinungsjahr
\item author - Autor des Buches
\end{itemize}
\end{frame}

\begin{frame}
\frametitle{BibTeX und \LaTeX ~in Kombination}
\begin{tabular}{|p{0.45\textwidth}|p{0.5\textwidth}|}
\hline
\color{nounibaredI}\textbackslash usepackage\color{black} ~natbib & Package natbib zur Darstellung eines alphabetischen LVZ notwendig\\
\hline
\color{nounibaredI}\textbackslash renewcommand\color{black}\{\color{nounibaredI}\textbackslash refname\color{black}\}\newline \{Literaturverzeichnis\} & Anpassung des Namens von Standard Literatur auf Literaturverzeichnis\\
\hline
\color{nounibaredI}\textbackslash bibliographystyle\color{black}\{alphadin\} & Auswahl des Style-Sheets zur Darstellung (hier das Style-Sheet \glqq alphadin\grqq)\\
\hline
\color{nounibaredI}\textbackslash bibliography\color{black}\{bibliography\} & Angabe der Einzubindenden BibTeX-Datei\\
\hline
\color{nounibaredI}\textbackslash addcontentsline\color{black}\{toc\}\{section\}\newline \{\color{nounibaredI}\textbackslash refname\color{black}\} & Aufnahme des LVZ ins Inhatsverzeichnis\\
\hline
\end{tabular}
\end{frame}

\begin{frame}
\frametitle{BibTeX und \LaTeX ~in Kombination}
\framesubtitle{Beispiele für Styles}
\begin{columns}
\begin{column}{.3\textwidth}
\begin{itemize}
\item \textbf{Alphadin-Style}
\vspace*{20mm}
\item \textbf{Abbrvdin-Style}
\end{itemize}
\end{column}
\begin{column}{.7\textwidth}
\image{\textwidth}{image/Styles1.png}{Alphadin und Abbrvdin Style}{img:Styles1}
\end{column}
\end{columns}
\end{frame}

\begin{frame}
\frametitle{BibTeX und \LaTeX ~in Kombination}
\framesubtitle{Beispiele für Styles cont'd}
\begin{columns}
\begin{column}{.3\textwidth}
\begin{itemize}
\item \textbf{Natdin-Style}
\vspace*{20mm}
\item \textbf{Plaindin-Style}
\end{itemize}
\end{column}
\begin{column}{.7\textwidth}
\image{\textwidth}{image/Styles2.png}{Natdin und Plaindin Style}{img:Styles2}
\end{column}
\end{columns}
\end{frame}

\begin{frame}
\frametitle{BibTeX und \LaTeX ~in Kombination}
\framesubtitle{Styles im fertigen Dokument}
\begin{columns}
\begin{column}{.3\textwidth}
\begin{itemize}
\item \textbf{Alphadin-Style}
\vspace*{10mm}
\item \textbf{Abbrvdin-Style}
\vspace*{10mm}
\item \textbf{Natdin-Style}
\vspace*{10mm}
\item \textbf{Plaindin-Style}
\end{itemize}
\end{column}
\begin{column}{.7\textwidth}
\vspace*{5mm}
\image{\textwidth}{image/Styles3.png}{Styles im fertigen Dokument}{img:Styles3}
\end{column}
\end{columns}
\end{frame}


\subsection*{Nützliches}

\begin{frame}
\frametitle{Nützliches}
\framesubtitle{LaTeX-Werkzeuge}
\begin{itemize}
  \item \textbf{Symbolliste}\footnote{\url{http://tug.ctan.org/info/symbols/comprehensive/symbols-a4.pdf}} -- Liste aller \LaTeX -Symbole \\
  \item \textbf{Detexify}\footnote{\url{http://detexify.kirelabs.org/classify.html}} -- \LaTeX -Symbolsuche\\
  \item \textbf{Tables Generator}\footnote{\url{http://www.tablesgenerator.com/}} -- Online-Generator für \LaTeX -Tabellen  \\
  \item \textbf{Wikibooks}\footnote{\url{https://en.wikibooks.org/wiki/LaTeX}} -- Dokumentation für viele \LaTeX -Anwendungsfälle \\
  \item \textbf{Overleaf}\footnote{\url{https://www.overleaf.com/}} -- Tutorials zu verschiedenen Themen \\
\end{itemize}
\end{frame}

%----------------------------------------------------------------------

\begin{frame}
\frametitle{Nützliches}
\framesubtitle{Pakete}
\begin{itemize}
  \item \textbf{paralist}\footnote{\url{https://www.ctan.org/pkg/paralist}} -- Aufzählungen ohne unnötige Abstände \\
  \item \textbf{parskip}\footnote{\url{https://www.ctan.org/pkg/parskip}} -- kleine Abstände zwischen Absätzen \\
  \item \textbf{booktabs}\footnote{\url{https://www.ctan.org/pkg/booktabs}} -- typographisch schöne Tabellen \\
  \item \textbf{coffee}\footnote{\url{http://www.hanno-rein.de/downloads/coffee.pdf}} -- falls man Kaffeeflecken auf dem Dokument braucht \\
  \item \textbf{ltablex}\footnote{\url{https://www.ctan.org/pkg/ltablex}} -- vereint gleich zwei nützliche Tabellentools \\
  \item \textbf{minted}\footnote{\url{https://www.ctan.org/pkg/minted}} -- Syntax-Highlighting für Quelltext (benötigt Python) \\
\end{itemize}
\end{frame}

%----------------------------------------------------------------------

\begin{frame}
\frametitle{Nützliches}
\framesubtitle{Pakete}
\begin{itemize}
  \item \textbf{forest}\footnote{\url{https://www.ctan.org/pkg/forest}} -- zeichnet (Binär-)Bäume \\
  \item \textbf{tikz}\footnote{\url{https://www.ctan.org/pkg/pgf}} -- \glqq{}TikZ ist kein Zeichenprogramm\grqq \\
  \item \textbf{tcolorbox}\footnote{\url{https://www.ctan.org/pkg/tcolorbox}} -- Boxen für Beispiele \\
  \item \textbf{pdfpages}\footnote{\url{https://www.ctan.org/pkg/pdfpages}} -- Einbinden von PDF-Dateien \\
  \item \textbf{subcaption}\footnote{\url{https://www.ctan.org/pkg/subcaption}} -- Bildunterschriften auch in Subfigures \\
  \item \textbf{phfnote}\footnote{\url{https://ctan.org/pkg/phfnote}} -- Kompaktes Layout für Mitschriften und Abgaben \\
\end{itemize}
\end{frame}

%----------------------------------------------------------------------

\begin{frame}
\frametitle{Nützliches}
\framesubtitle{Pakete}
\begin{itemize}
  \item \textbf{todonotes}\footnote{\url{https://www.ctan.org/pkg/todonotes}} -- ToDo-Markierungen und Table of ToDos \\
  \item \textbf{cmbright}\footnote{\url{https://www.ctan.org/pkg/cmbright}} -- Serifenlose Schriften für \LaTeX \\
  \item \textbf{colortbl}\footnote{\url{https://www.ctan.org/pkg/colortbl}} -- Farbige \LaTeX -Tabellen \\
  \item \textbf{bbcard}\footnote{\url{https://www.ctan.org/pkg/bbcard}} -- Bullshit-Bingo-Karten \\
\end{itemize}
\end{frame}

%----------------------------------------------------------------------

\begin{frame}
\frametitle{Nützliches}
\framesubtitle{Eigene Befehle}
\begin{columns}
\hspace*{4.7mm}
\begin{column}{0.5\textwidth}
\textbf{Definition:}\\
\end{column}
\begin{column}{0.5\textwidth}
\textbf{Benutzung:}\\
\end{column}
\end{columns}
\bigskip
\begin{columns}
\hspace*{4.7mm}
\begin{column}{0.5\textwidth}
\begin{ttfamily}{\normalsize
\color{nounibaredI}\textbackslash newcommand\color{black}\{\textbackslash vektor\}[2]\{\\
\color{unibablueI}\textbackslash begin\color{black}\{pmatrix\}\\
\color{unibayellowI}\# 1 \color{nounibaredI}\textbackslash \textbackslash\\
\color{unibayellowI} \# 2\\
\color{unibablueI}\textbackslash end\color{black}\{pmatrix\}\\
\}\\
}
\end{ttfamily}
\end{column}
\begin{column}{0.5\textwidth}
\begin{ttfamily}{\normalsize
\color{unibayellowI}\$ \color{nounibaredI}\textbackslash vektor\color{black}\{3\}\{-2\} \color{unibayellowI}\$ \\}
\end{ttfamily}
\medskip
$
\begin{pmatrix}
3 \\ -2
\end{pmatrix}
$
\end{column}
\end{columns}
\bigskip
Weitere Infos zu vielen \LaTeX -Paketen findet ihr bei Wikibooks zu \LaTeX\footnote{http://en.wikibooks.org/wiki/LaTeX}.\\
\end{frame}


%\input{content/beamerclass}



%%%%%%%%%%%%%%%%%%%%%%%%%%%%%%%%%%%%%%%%%
%%%%%%%%%% References          %%%%%%%%%%
%%%%%%%%%%%%%%%%%%%%%%%%%%%%%%%%%%%%%%%%%
%\section*{}
%\begin{frame}[allowframebreaks]{References}
%\def\newblock{\hskip .11em plus .33em minus .07em}
%\scriptsize
%\bibliographystyle{IEEEtran}
%\bibliography{literature/bib}
%\normalsize
%\end{frame}




%% Last frame
\frame{
  \vspace{2cm}
  {\huge Thank you!}

  \vspace{20mm}
  \nocite*
  \vspace{0mm}

  \begin{flushright}

  Fachschaft WIAI

    \structure{\footnotesize{\href{mailto:fachschaft.wiai@uni-bamberg.de}{fachschaft.wiai@uni-bamberg.de}}}

  \end{flushright}

}


\end{document}
