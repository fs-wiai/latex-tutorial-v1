
\begin{frame}
\frametitle{Referenzen}
\framesubtitle{Abbildungen einfügen – A closer look}
\begin{tabbing}
\begin{ttfamily}\color{nounibaredI}\textbackslash label\color{black}\{Schl"ussel\}\end{ttfamily} \= Mit diesem Befehl weist man einen Schl"ussel zu. Später\\
\> im Text kann man dann durch eine Referenz auf diese\\
\> Stelle verweisen.\\
\> Dies geschieht mit dem Befehl
\begin{ttfamily}\color{nounibaredI}\textbackslash ref\color{black}\{Schl"ussel\}\end{ttfamily}.
\end{tabbing}
\begin{ttfamily}Der kleine Tux ist ein Allesfresser. Egal ob Gem"use oder
Schnittlauch, nichts ist vor ihm sicher (siehe Abb.
\color{nounibaredI}\textbackslash ref\color{black}\{img:kleinertux\}).\end{ttfamily}\\[3mm]
\textbf{Ergebnis:}\\[3mm]
\begin{minipage}{\textwidth}\begin{rm}
Der kleine Tux ist ein Allesfresser. Egal ob Gem"use oder
Schnittlauch, nichts ist vor ihm sicher (siehe Abb.
1).\end{rm} \end{minipage}\\[3mm]
%\begin{exampleblock}{Und warum das Ganze?}
%Durch solche Referenzen wird immer auf das richtige Bild verwiesen, auch wenn zwischendurch noch weitere Bilder einfügt wurden.
%\end{exampleblock}
\textbf{Und warum das Ganze?}\\
Durch solche Referenzen wird immer auf das richtige Bild verwiesen, auch wenn zwischendurch noch weitere Bilder einfügt wurden.
\end{frame}

%-------------------------------------

\begin{frame}
\frametitle{Referenzen}
\framesubtitle{cref}

Mit dem Package \textit{cleveref} wird die Referenz direkt in der richtigen Sprache beschriftet.\\[3mm]

\begin{ttfamily}siehe 
\color{nounibaredI}\textbackslash cref\color{black}\{img:tux1\}\end{ttfamily}\\[3mm]
\textbf{Ergebnis:}\\[3mm]

\begin{rm}
siehe Abb. 1 \end{rm}\\[3mm]

\medskip

\begin{block}{Hinweis}
\begin{ttfamily}\color{nounibaredI}\textbackslash cref\color{black}\{sec:intro\}\end{ttfamily} funktioniert auch mit Verweisen auf Sections, Tabellen usw\ldots ~Auch sie m"ussen mit Schl"usseln versehen werden: \begin{ttfamily}\color{nounibaredI}\textbackslash section\color{black}\{Introduction\color{nounibaredI}\textbackslash label\color{black}\{sec:intro\}\}\end{ttfamily}\\[3mm] %\label{sec:Introduction}
Schreibt man \begin{ttfamily}\color{nounibaredI}\textbackslash Cref\color{black}\{img:tux1\}\end{ttfamily} groß, wird der Begriff mit einem Großbuchstaben begonnen.
\end{block}

\end{frame}
