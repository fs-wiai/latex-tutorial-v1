\section{Intro}

\subsection{Anwendungsbereiche, Sinn \& Zweck}
\begin{frame}[t]
\frametitle{Einführung}
\framesubtitle{Sinn -- Unsinn -- Wahnsinn}
\bigskip
\bigskip
\bigskip

\begin{columns}[t]
\begin{column}{.3\textwidth}
\textbf{Sinnvoll}\\[3mm]
\begin{itemize}
\item wissenschaftliche Arbeiten
\item Bücher
\item Lebenslauf \& Bewerbungen
\end{itemize}
\end{column}
\begin{column}{.30\textwidth}
\textbf{Unsinn}\\[3mm]
\begin{itemize}
\item private Briefe
\item Geburtstags-einladungen
\item Getr"ankekarten
\end{itemize}
\end{column}
\begin{column}{.3\textwidth}
\textbf{Wahnsinn}\\[3mm]
\begin{itemize}
\item Einkaufszettel
\item Brainstorming
\item \ldots
\end{itemize}
\end{column}
\end{columns}
\end{frame}

%-------------------------------------------------------------------------------

\subsection{Vorteile \& Nachteile}
\begin{frame}
\frametitle{Einführung}
\framesubtitle{Vorteile \& Nachteile}
\textbf{Vorteile}
\begin{itemize}
\item  automatisch generierte Abbildungs- und Inhaltsverzeichnisse
\item  vorgefertigte Layouts
\item  verschiedene Dokumenttypen (Briefe, Präsentationen etc.)
\item  Aufteilen in Unterdokumente möglich
\item  Versionsverwaltung möglich (wird hier \textbf{nicht} vorgestellt)
\item  viele Erweiterungen (z.B. für Syntax-Highlighting)
\end{itemize}

\textbf{Nachteile}
\begin{itemize}
\item  Resultat nicht sofort ersichtlich
\item  viele, zum Teil komplexe Befehle
\item  erhebliche Unterschiede zu Textverarbeitungsprogrammen wie Word/OpenOffice/LibreOffice
\end{itemize}
\end{frame}


%-------------------------------------------------------------------------------

\begin{frame}
\frametitle{Vom Code zum Dokument}
\framesubtitle{Kein WYSIWYG}
\begin{columns}
\begin{column}{.7\textwidth}
\image{\textwidth}{image/worddoc.jpg}{\textbf{W}hat \textbf{Y}ou \textbf{S}ee \textbf{I}s \textbf{W}hat \textbf{Y}ou
\textbf{G}et}{img:worddoc}
\end{column}
\begin{column}{.3\textwidth}
\image{\textwidth}{image/codescreen.png}{What Will I Get?}{img:codescreen}
%% Compile Animation
\end{column}
\end{columns}
\end{frame}

%-------------------------------------------------------------------------------

\begin{frame}
\frametitle{Einführung}
\framesubtitle{Vorgehensweise}
\begin{columns}[onlytextwidth]
\begin{column}{0.40\textwidth}
\image{.8\textwidth}{image/codescreen.png}{Textdatei mit \LaTeX-Code}{img:code}
\end{column}
\begin{column}{0.25\textwidth}
\image{.8\textwidth}{image/miktex.jpg}{Compiler (z.B. MikTeX)}{img:miktex}
\end{column}
\begin{column}{0.25\textwidth}
\image{.6\textwidth}{image/pdflogo.png}{Sch\"ones, lesbares und druckbares Dokument}{img:pdf}
\end{column}
\end{columns}
\end{frame}



%-------------------------------------------------------------------------------

\subsection{\LaTeX --Compiler}
\begin{frame}
\frametitle{Einführung}
\framesubtitle{\LaTeX - Compiler}
\begin{columns}[t]
\begin{column}{.4\textwidth}
\textbf{Windows:}\\
\begin{itemize}
  \item MikTex (\url{https://miktex.org})\\
   2 Varianten: Basic oder Complete
  \item ProTeXt (\url{https://www.tug.org/protext})\\
   enthält MikTex, TeXnicCenter und Ghostscript – einfache Installation\\
\end{itemize}
\end{column}
\begin{column}{.6\textwidth}
\textbf{Unix}
\begin{itemize}
  \item \textbf{Linux:} TeXLive\\
Pakete unter Ubuntu: {\ttfamily texlive-full} ist das Meta-Paket mit allen
ben\"otigten Paketen. Enthält auch folgende:
\begin{itemize}
  \item {\ttfamily texlive-base
  \item texlive-lang-german}
\end{itemize}
Installation: {\ttfamily sudo apt-get install texlive-full}
\item \textbf{MacOS:} MacTeX (\url{https://www.tug.org/mactex/})\\
\end{itemize}
\end{column}
\end{columns}
\end{frame}

%-------------------------------------------------------------------------------

%\subsection{Freie Editoren}
%
%\subsubsection{*nix}
%\begin{frame}
%\frametitle{Einführung}
%\framesubtitle{Freie Editoren -- Linus \& MacOS }
%\begin{itemize}
% \item Kile\footnote{http://kile.sourceforge.net/}\\KDE-Programm, auch unter Gnome\slash Unity etc.
% verwendbar. Installation auf Debiansystemen mit {\ttfamily sudo apt-get
% install kile}.
%  \item Vim \LaTeX -suite (Plugin)\footnote{http://vim-latex.sourceforge.net/}\\
%  Ein Traum f"ur Vim-User.
%  \item TexShop (MacOS)\footnote{http://pages.uoregon.edu/koch/texshop/}
%\end{itemize}
%\end{frame}

%-------------------------------------------------------------------------------

%\subsubsection{Windows}
%\begin{frame}
%\frametitle{Einführung}
%\framesubtitle{Freie Editoren -- Windows}
%\begin{itemize}
%\item TeXnicCenter\footnote{http://www.texniccenter.org/}
%  %\item \ldots
%\end{itemize}
%\end{frame}

%-------------------------------------------------------------------------------

\subsubsection{Cross-Platform}
\begin{frame}
\frametitle{Einführung}
\framesubtitle{Freie Editoren -- Cross-Platform}
\begin{itemize}
  \item TeXstudio\footnote{\url{http://texstudio.sourceforge.net/}}\\
  Sehr solide und mächtig, verwenden wir hier im Tutorium
  \item TeXMaker\footnote{\url{http://www.xm1math.net/texmaker}}\\
   Grundlage von TeXstudio, etwas schlanker mit weniger Funktionen
  \item TeXlipse\footnote{\url{http://texlipse.sourceforge.net}}\\ Für fortgeschrittene User, Plugin für
  Eclipse, gute IDE-Unterstützung, Code-Completion, Autobuilds, Versionsverwaltung etc.
\end{itemize}
\end{frame}

%-------------------------------------------------------------------------------

\subsubsection{In der Cloud}
\begin{frame}
\frametitle{Online}
\framesubtitle{Multiediting -- auch Kollaborativ}
\begin{itemize}
  \item Overleaf\footnote{\url{https://www.overleaf.com}}\\
  Für kleine Projekte (60 Dateien) kostenlos
  \item ShareLatex\footnote{\url{https://www.sharelatex.com}, \url{https://github.com/sharelatex}}\\
  Teilweise Open Source, kostenlos für Einzelbenutzer, zuletzt von Overleaf aufgekauft
  \item FlyLatex\footnote{\url{https://github.com/alabid/flylatex}}\\
  Open Source, weniger mächtig
  \item Viele Weitere\ldots
  \vspace{7mm}
  \item In der Uni wird auch oft Git\footnote{\url{https://git-scm.com}} zur Synchronisation verwendet.
\end{itemize}
\end{frame}


%-------------------------------------------------------------------------------

\begin{frame}
%TODO texstudio!
\frametitle{TeXstudio}
\framesubtitle{\"Uberblick}
\image{\textwidth}{image/texstudio_overview.png}{Das Standardfenster von TeXstudio}{img:texstudio1}
\end{frame}

%-------------------------------------------------------------------------------

%\begin{frame}
%\frametitle{TeXmaker}
%\framesubtitle{Synctex}
%\image{\textwidth}{image/synctex.png}{Synctex}{img:synctex}

%\end{frame}

\begin{frame}
\frametitle{Mein erstes \LaTeX~-Dokument}
\begin{block}{Neue Befehle:}
\begin{itemize}
\item \begin{ttfamily}\color{nounibaredII}\textbackslash documentclass\color{nounibagreenI}\color{black}\{article\}\end{ttfamily}
\item \begin{ttfamily}\color{unibablueI}\textbackslash begin\color{black}\{document\}\end{ttfamily}
\item \begin{ttfamily} Inhalt als plain text \end{ttfamily}
\item \begin{ttfamily}\color{unibablueI}\textbackslash end\color{black}\{document\}\end{ttfamily}
\end{itemize}
\end{block}
Das ist alles, was man für ein \LaTeX -Dokument braucht. Und das probieren wir jetzt aus!

\end{frame}
