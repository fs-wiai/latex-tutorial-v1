\section{Aufgabe 4}
\subsection{Tabellen \& Formeln}
Die Tabelle \ref{tab:ani} besteht aus 3 Spalten:\\
Die erste Spalte ist mit einem p von 25mm definiert. Die zweite und die dritte Spalte sind zentriert.

\begin{longtable}{p{25mm}|c|c}
& Fuchs & Elster\\
\hline
\hline
Familie & Hunde & Rabenvögel\\
\hline
Gewicht & m: 6,6kg w: 5,5kg & 200--250g\\
\hline
Geschwindigkeit $ = \sqrt{v\cdot v}$ & $55\frac{km}{h}$ & mind. superschnell: $\lim\limits_{x \rightarrow \infty} x \cdot v$ \\
\hline
Farbe & tödlich & schwarz\\
\hline
\caption{Wild Animals}
\label{tab:ani}
\end{longtable}

Der Sinn des Lebens$^2$: $\prod\limits_{i=1}^{n+1} i + \sum_{j=0}^{n} j \cdot \int\limits_{\pi}^{Daumen} 42$

\subsection{Aufzählungen}
Um bei den vielen Verschachtelungen nicht den Überblick zu verlieren, sind Einrückungen der items sinnvoll.
\begin{enumerate}
  \item 
  \begin{enumerate}
    \item Vorteile des Fuchses:
    \item
    \begin{itemize}
      \item schlau
      \item schaut cool aus
    \end{itemize}
    \item Nachteile des Fuchses:
    \begin{itemize}
      \item Pelz wird verarbeitet
      \item sehr viele Autos fahren gerne über Füchse
    \end{itemize}
    \item Spam Spam Spam
  \end{enumerate}
  \item
  \begin{enumerate}
    \item Vorteile der Elster...
    \item Nachteile der Elster:
    \begin{itemize}
      \item Diebischkeit wird bestraft
      \item viele landen hinter Gittern
    \end{itemize}
      \item singt ganz gut, aber ist gefährlich
  \end{enumerate}
\end{enumerate}

\subsection{(Un)Logik}
\begin{itemize}
	\item $\lnot\forall x \Leftrightarrow \{\exists x\}$
	\item $[\exists xPx] \rightarrow \forall x \lnot Px$
\end{itemize}

\subsection{Code}
\textit{Hello World} in Java:\\
\lstset{language=Java, commentstyle=\color{green}}
\begin{lstlisting}
  public class Hello{
      public static void main(String[] args){
      
         //Hier wird der Text ausgegeben:
         System.out.println("Hello World!");
      }
  }
\end{lstlisting}