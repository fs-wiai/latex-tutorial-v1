\section{Graphics}
\begin{frame}
\frametitle{Graphics}
\framesubtitle{Use Of Graphics} 
\begin{exampleblock}{New packages in this section}
\begin{itemize}
\item graphicx 
\item float
\end{itemize}
\end{exampleblock}

\begin{block}{New commands in this section}
\begin{itemize}
\item \color{nounibaredI}\textbackslash includegraphics\color{black}\{File\}
\item \color{nounibaredI}\textbackslash caption\color{black}\{caption\}
\item \begin{ttfamily}\color{nounibaredI}\textbackslash label\color{black}\{label\}
\item \color{nounibaredI}\textbackslash ref\color{black}\{reference\}\end{ttfamily}
\end{itemize}
\end{block}

\end{frame}

%-------------------------------------------------------------------------------

\begin{frame}
\frametitle{Graphics}
\framesubtitle{Insert Graphics}
\begin{tabbing}
\textbackslash includegraphics[option]\{File\}xx\=\=\=\kill
\begin{ttfamily}
\color{unibablueI}\textbackslash begin\color{black}\{figure\}\color{nounibagreenI}[option]\color{black}
\end{ttfamily}
\>\>\textbf{Possible options for positioning:}\\
\>\>\begin{ttfamily}\color{nounibagreenI}[h]\color{black}\end{ttfamily} = At this very position\\
\>\>\begin{ttfamily}\color{nounibagreenI}[t]\color{black}\end{ttfamily} = On top of the page\\
\>\>\begin{ttfamily}\color{nounibagreenI}[b]\color{black}\end{ttfamily} = On bottom of the page\\
\>\>\begin{ttfamily}\color{nounibagreenI}[p]\color{black}\end{ttfamily} = Positioning on an own
page\\[5mm]
~\\[5mm]
\color{nounibaredI}\textbackslash includegraphics\color{nounibagreenI}[option]\color{black}\{File\}
\>\>\textbf{Possible options for insertion:}\\
\>\>\begin{ttfamily}\color{nounibagreenI}[width=300pt]\color{black}\end{ttfamily}= scale to a width\\
\>\>\begin{ttfamily}\color{nounibagreenI}[height=5cm]\color{black}\end{ttfamily}= scale to a height\\
\>\>scale, angle and many more\ldots\\
\>\>Combiantions possible:\\
\>\>\begin{ttfamily}\color{nounibagreenI}[width=\textbackslash textwidth,height=5cm]\color{black}\end{ttfamily}
\end{tabbing}
\end{frame}

%-------------------------------------------------------------------------------

\begin{frame}
\frametitle{Graphics}
\framesubtitle{Positioning Of Figures}
\begin{columns}
\begin{column}{.5\textwidth}
{\ttfamily {\footnotesize
\color{nounibaredI}\color{nounibaredI}\textbackslash documentclass\color{black}\{article\} \\
\color{nounibaredI}\color{nounibaredI}\textbackslash usepackage\color{black}\{graphicx\} \\
\color{nounibaredI}\color{unibablueI}\textbackslash\color{unibablueI}begin\color{black}\color{black}\{document\} \\
\color{nounibaredI}\color{unibablueI}\textbackslash\color{unibablueI}begin\color{black}\color{black}\{figure\}\color{nounibagreenI}[h]\color{black} \\
\color{nounibaredI}\color{unibablueI}\textbackslash\color{unibablueI}begin\color{black}\color{black}\{center\} \\
	\color{nounibaredI}\color{nounibaredI}\textbackslash includegraphics\color{black}\color{nounibagreenI}[width=50mm]\color{black}\{tux.png\} \\
\color{nounibaredI}\color{nounibaredI}\textbackslash caption\color{black}\{Tiny Tux\} \\
\color{nounibaredI}\color{nounibaredI}\textbackslash label\color{black}\{img:tinytux\} \\
\color{nounibaredI}\color{unibablueI}\textbackslash\color{unibablueI}end\color{black}\color{black}\{center\} \\
\color{nounibaredI}\color{unibablueI}\textbackslash\color{unibablueI}end\color{black}\color{black}\{figure\} \\
\color{nounibaredI}\color{unibablueI}\textbackslash\color{unibablueI}end\color{black}\color{black}\{document\} \\
}}

\begin{alertblock}{Attention!}
\color{nounibaredI}\textbackslash label\color{black}\{\} always after \color{nounibaredI}\textbackslash caption\color{black}\{\}
\end{alertblock}
\end{column}

\begin{column}{.5\textwidth} 
\begin{figure}
\begin{center}
    \includegraphics[width=\textwidth]{image/tux.png}
\caption{Tiny Tux}
\label{img:kleinertux}
\end{center}
\end{figure}
\end{column}
\end{columns}
\end{frame}

%-------------------------------------------------------------------------------

\begin{frame}[t]
\medskip
\frametitle{Graphics}
\framesubtitle{Positioning Of Figures II}
Despite the positioning, the figure often gets out of place, because it is not always possible to insert it at a useful position.\\
\textbf{Solution:} The package {\ttfamily float} ensures a better positioning in most cases.

\begin{columns}
\begin{column}{.5\textwidth}
{\ttfamily {\footnotesize
\color{nounibaredI}\color{nounibaredI}\textbackslash documentclass\color{black}\{article\} \\
\color{nounibaredI}\color{nounibaredI}\textbackslash usepackage\color{black}\{graphicx\} \\
\color{nounibaredI}\color{nounibaredI}\textbackslash usepackage\color{black}\{float\} \\
\color{nounibaredI}\color{unibablueI}\textbackslash\color{unibablueI}begin\color{black}\color{black}\{document\} \\
\color{nounibaredI}\color{unibablueI}\textbackslash\color{unibablueI}begin\color{black}\color{black}\{figure\}\color{nounibagreenI}[H]\color{black} \\
\color{nounibaredI}\color{unibablueI}\textbackslash\color{unibablueI}begin\color{black}\color{black}\{center\} \\
	\color{nounibaredI}\color{nounibaredI}\textbackslash includegraphics\color{black}\color{nounibagreenI}[width=70mm]\color{black}\{pfad/tux.png\} \\
\color{nounibaredI}\color{nounibaredI}\textbackslash caption\color{black}\{Der kleine Tux jetzt in Float\} \\
\color{nounibaredI}\color{nounibaredI}\textbackslash label\color{black}\{img:kleinertux-float\} \\
\color{nounibaredI}\color{unibablueI}\textbackslash\color{unibablueI}end\color{black}\color{black}\{center\} \\
\color{nounibaredI}\color{unibablueI}\textbackslash\color{unibablueI}end\color{black}\color{black}\{figure\} \\
\color{nounibaredI}\color{unibablueI}\textbackslash\color{unibablueI}end\color{black}\color{black}\{document\} \\
}}
\end{column}

\begin{column}{.5\textwidth} 
\begin{figure}
\begin{center}
    \includegraphics[width=35mm]{image/tux.png}
\caption{Tiny Tux in Float}
\label{img:kleinertux_float}
\end{center}
\end{figure}
\end{column}
\end{columns}

\end{frame}
