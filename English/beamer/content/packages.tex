
\section{Formatting}

\begin{frame}
\frametitle{A First Application Example}
\framesubtitle{Headlines, Table of Contents, simple formatting,
special characters}
\begin{block}{New commands in this section}
\begin{multicols}{2}
\begin{itemize}
  \item \begin{ttfamily}\color{nounibaredII}\textbackslash usepackage\color{black}\{package\}
  \item \color{nounibaredII}\textbackslash command\color{nounibagreenI}[poss\_options]\color{black}\{\\formatted\_text\}
  \item \color{unibablueI}\textbackslash begin\color{black}\{environment\}
  \item \color{unibablueI}\textbackslash end\color{black}\{environment\}
  \item \color{nounibaredII}$\backslash\backslash$\color{black}
  \item \color{nounibaredI}\textbackslash newpage\color{black}
  \item \color{unibablueI}\textbackslash sub$^*$section\color{black}\{Title\}
  \item $\color{nounibaredII}\backslash$\color{nounibaredII}textbf\color{black}\{Text\}
  \item $\color{nounibaredII}\backslash$\color{nounibaredII}textit\color{black}\{Text\}
  \item $\color{nounibaredII}\backslash$\color{nounibaredII}underline\color{black}\{Text\}
  \item \color{nounibaredI}$\color{nounibaredI}\backslash$tiny
  \item \color{nounibaredI}$\color{nounibaredI}\backslash$scriptsize
  \item \color{nounibaredI}$\color{nounibaredI}\backslash$footnotesize
  \item \color{nounibaredI}$\color{nounibaredI}\backslash$normalsize
  \item \color{nounibaredI}$\color{nounibaredI}\backslash$large
  \item \color{nounibaredI}$\color{nounibaredI}\backslash$Large
  \item \color{nounibaredI}$\color{nounibaredI}\backslash$LARGE
  \item \color{nounibaredI}$\color{nounibaredI}\backslash$huge\end{ttfamily}
\end{itemize}
\end{multicols}
\end{block}
\end{frame}


%\begin{frame}
%\frametitle{Ein erstes Anwendungsbeispiel}
%\framesubtitle{Pakete einbinden und Befehle anwenden}
%\begin{itemize}
%  \item Pakete sind Sammlungen von Befehlen oder enthalten z.B. Zeichensätze.\\ Sie werden zu
%  Beginn einer \TeX-Datei angegeben:\\
%  \smallskip
%\textbf{\begin{ttfamily}\color{nounibaredII}\textbackslash usepackage\color{black}\{babel\}
%
%\smallskip
%\end{ttfamily}}
% Einbinden des Paketes „\begin{ttfamily}babel\end{ttfamily}“. (F\"ur Internationalisierung)
%\item Schreibweise von Latex-Befehlen:
%
%\textbf{\begin{ttfamily}\color{nounibaredII}\textbackslash befehl\color{nounibagreenI}[evtl\_optionen]\color{black}\{Formatierter\_Text\}\end{ttfamily}}
%\begin{itemize}
%  \item in \begin{ttfamily}\{\}\end{ttfamily} stehen immer notwendige Parameter bzw. Text
% \item in \begin{ttfamily}[ ]\end{ttfamily} stehen (falls vorhanden)
% zus"atzliche, optionale Parameter
% \item zum Beispiel:
%
%
%\begin{ttfamily}
%\color{nounibaredII}\textbackslash documentclass\color{nounibagreenI}[a4paper,12pt,pdftex,ngerman]\color{black}\{article\}
%\end{ttfamily}
%\end{itemize}
%\end{frame}

\begin{frame}
\frametitle{A First Application Example}
\framesubtitle{Commands cont'd}
\begin{columns}
\begin{column}{0.6\textwidth}
\begin{ttfamily}\scriptsize
\color{nounibaredI}\color{nounibaredI}\textbackslash documentclass\color{black}\color{nounibagreenI}[a4paper, pdftex, ngerman]\color{black}\{article\} \\
\color{nounibaredI}\color{nounibaredI}\textbackslash usepackage\color{black}\color{nounibagreenI}[utf8]\color{black}\{inputenc\} \\
\color{nounibaredI}\color{nounibaredI}\textbackslash usepackage\color{black}\color{nounibagreenI}[T1]\color{black}\{fontenc\} \\
\color{nounibaredI}\color{nounibaredI}\textbackslash usepackage\color{black}\{babel\} \\
\color{nounibaredI}\color{unibablueI}\textbackslash\color{unibablueI}begin\color{black}\color{black}\{document\} \\
Das ist ein einfaches Minidokument \\
ohne Besonderheiten. Zeilenumbrüche \\
funktionieren immer automatisch! \\
Mehrere \\
Leerzeichen hintereinander werden  \\
zu einem zusammengefasst. \\
Getrennt wird auch automatisch.\color{nounibaredI}\color{nounibaredI}\textbackslash \color{nounibaredI}\textbackslash \color{black} \\
Mit zwei Backslashs beginnt eine neue \\
Zeile.\color{nounibaredI}\color{nounibaredI}\textbackslash \color{nounibaredI}\textbackslash \color{black} \\
Ein neuer Absatz entsteht durch eine \\
leere Zeile. \\
\color{nounibaredI}\color{unibablueI}\textbackslash\color{unibablueI}end\color{black}\color{black}\{document\} \\

\end{ttfamily}
\end{column}

\begin{column}{0.4\textwidth}
There are different types of documents.\\ The document type
\begin{ttfamily}article\end{ttfamily} is used here (also possible:
\begin{ttfamily}book\end{ttfamily} and \begin{ttfamily}report\end{ttfamily}). The input in
\begin{ttfamily}[]\end{ttfamily} indicates paper size and font size of the
standard text.\\
%! TODO!
\end{column}
\end{columns}
\end{frame}

\begin{frame}
\frametitle{A First Application Example}
\framesubtitle{Commands cont'd}
\begin{columns}
\begin{column}{0.6\textwidth}
\begin{ttfamily}\scriptsize
\color{nounibaredI}\color{nounibaredI}\textbackslash documentclass\color{black}\color{nounibagreenI}[a4paper, pdftex, ngerman]\color{black}\{article\} \\
\color{nounibaredI}\color{nounibaredI}\textbackslash usepackage\color{black}\color{nounibagreenI}[utf8]\color{black}\{inputenc\} \\
\color{nounibaredI}\color{nounibaredI}\textbackslash usepackage\color{black}\color{nounibagreenI}[T1]\color{black}\{fontenc\} \\
\color{nounibaredI}\color{nounibaredI}\textbackslash usepackage\color{black}\{babel\} \\
\color{nounibaredI}\color{unibablueI}\textbackslash\color{unibablueI}begin\color{black}\color{black}\{document\} \\
Das ist ein einfaches Minidokument \\
ohne Besonderheiten. Zeilenumbrüche \\
funktionieren immer automatisch! \\
Mehrere \\
Leerzeichen hintereinander werden  \\
zu einem zusammengefasst. \\
Getrennt wird auch automatisch.\color{nounibaredI}\color{nounibaredI}\textbackslash \color{nounibaredI}\textbackslash \color{black} \\
Mit zwei Backslashs beginnt eine neue \\
Zeile.\color{nounibaredI}\color{nounibaredI}\textbackslash \color{nounibaredI}\textbackslash \color{black} \\
Ein neuer Absatz entsteht durch eine \\
leere Zeile. \\
\color{nounibaredI}\color{unibablueI}\textbackslash\color{unibablueI}end\color{black}\color{black}\{document\} \\

 \normalsize
\end{ttfamily}
\end{column}
\begin{column}{0.4\textwidth}
\begin{ttfamily}\textbf{\color{unibablueI}\textbackslash begin\color{black}\{environment\}}\end{ttfamily}\\
A new environment begins, in this case the actual document.\\[5mm]

\begin{ttfamily}\textbf{\color{unibablueI}\textbackslash end\color{black}\{environment\}}\end{ttfamily}\\
The environment that started with \begin{ttfamily}\textbf{\color{unibablueI}\textbackslash begin}\color{black}\{\}\end{ttfamily}
ends here.\\[5mm]

\begin{ttfamily}\textbf{\color{nounibaredII}$\backslash\backslash$}\color{black}
~line break\end{ttfamily}\\
\end{column}
\end{columns}
\end{frame}



\begin{frame}
\frametitle{A First Application Example}
\framesubtitle{Packages}
\begin{columns}
\begin{column}{0.6\textwidth}
\begin{ttfamily}\scriptsize
\color{nounibaredI}\color{nounibaredI}\textbackslash documentclass\color{black}\color{nounibagreenI}[a4paper, pdftex, ngerman]\color{black}\{article\} \\
\color{nounibaredI}\color{nounibaredI}\textbackslash usepackage\color{black}\color{nounibagreenI}[utf8]\color{black}\{inputenc\} \\
\color{nounibaredI}\color{nounibaredI}\textbackslash usepackage\color{black}\color{nounibagreenI}[T1]\color{black}\{fontenc\} \\
\color{nounibaredI}\color{nounibaredI}\textbackslash usepackage\color{black}\{babel\} \\
\color{nounibaredI}\color{unibablueI}\textbackslash\color{unibablueI}begin\color{black}\color{black}\{document\} \\
Das ist ein einfaches Minidokument \\
ohne Besonderheiten. Zeilenumbrüche \\
funktionieren immer automatisch! \\
Mehrere \\
Leerzeichen hintereinander werden  \\
zu einem zusammengefasst. \\
Getrennt wird auch automatisch.\color{nounibaredI}\color{nounibaredI}\textbackslash \color{nounibaredI}\textbackslash \color{black} \\
Mit zwei Backslashs beginnt eine neue \\
Zeile.\color{nounibaredI}\color{nounibaredI}\textbackslash \color{nounibaredI}\textbackslash \color{black} \\
Ein neuer Absatz entsteht durch eine \\
leere Zeile. \\
\color{nounibaredI}\color{unibablueI}\textbackslash\color{unibablueI}end\color{black}\color{black}\{document\} \\

 \normalsize
\end{ttfamily}
\end{column}
\begin{column}{0.4\textwidth}
\begin{ttfamily}\textbf{ngerman}\end{ttfamily}\\
The typical formattings and rules of the german language are used (instead of \grqq ngerman\grqq ~you can use \grqq english\grqq ~for english texts).\\[5mm]

\begin{ttfamily}\textbf{inputenc}\end{ttfamily}\\
Defines the character-\\set, that has to be used. You should always use
\begin{ttfamily}UTF-8\end{ttfamily}, because it runs
on all operating systems.\\
\end{column}
\end{columns}
\end{frame}

\begin{frame}
\frametitle{Excursion}
\framesubtitle{Character encoding}
\begin{columns}
\begin{column}{0.6\textwidth}
\image{\textwidth}{image/utf8.png}{UTF-8 in Texmaker}{img:utf8}

\end{column}
\begin{column}{0.4\textwidth}
When you open a document, the correct character set is used automatically.
When you create a new document, it is saved with the default settings 
 of the editor. In the Texmaker settings, the same character set has to be used, that is used in the LaTeX-document.\\
\end{column}
\end{columns}
\textbf{When working in a team, every team member has to set \underline{UTF-8} in
the editor, or problems ar inevitable!} (Broken
vowel mutations, errors during compiling and much more, if there are not only „Windows“-users.)
\end{frame}

\begin{frame}
\frametitle{A First Application Example}
\framesubtitle{As .PDF}
\begin{columns}
\begin{column}{0.5\textwidth}
\begin{ttfamily}\scriptsize
\color{nounibaredI}\color{nounibaredI}\textbackslash documentclass\color{black}\color{nounibagreenI}[a4paper, pdftex, ngerman]\color{black}\{article\} \\
\color{nounibaredI}\color{nounibaredI}\textbackslash usepackage\color{black}\color{nounibagreenI}[utf8]\color{black}\{inputenc\} \\
\color{nounibaredI}\color{nounibaredI}\textbackslash usepackage\color{black}\color{nounibagreenI}[T1]\color{black}\{fontenc\} \\
\color{nounibaredI}\color{nounibaredI}\textbackslash usepackage\color{black}\{babel\} \\
\color{nounibaredI}\color{unibablueI}\textbackslash\color{unibablueI}begin\color{black}\color{black}\{document\} \\
Das ist ein einfaches Minidokument \\
ohne Besonderheiten. Zeilenumbrüche \\
funktionieren immer automatisch! \\
Mehrere \\
Leerzeichen hintereinander werden  \\
zu einem zusammengefasst. \\
Getrennt wird auch automatisch.\color{nounibaredI}\color{nounibaredI}\textbackslash \color{nounibaredI}\textbackslash \color{black} \\
Mit zwei Backslashs beginnt eine neue \\
Zeile.\color{nounibaredI}\color{nounibaredI}\textbackslash \color{nounibaredI}\textbackslash \color{black} \\
Ein neuer Absatz entsteht durch eine \\
leere Zeile. \\
\color{nounibaredI}\color{unibablueI}\textbackslash\color{unibablueI}end\color{black}\color{black}\{document\} \\

 \normalsize
\end{ttfamily}
\end{column}

\begin{column}{0.5\textwidth}
\image{\textwidth}{image/minidocument.png}{The code of the left side as .pdf.}{listing:minidocument}
\end{column}
\end{columns}
\end{frame}



\begin{frame}
\frametitle{Sections}
\framesubtitle{Chapters}
\begin{columns}
\begin{column}{0.5\textwidth}
\begin{ttfamily}\scriptsize
\color{nounibaredI}\color{nounibaredI}\textbackslash documentclass\color{black}\color{nounibagreenI}[a4paper, pdftex, 12pt, ngerman]\color{black}\{article\} \\
\color{nounibaredI}\color{nounibaredI}\textbackslash usepackage\color{black}\color{nounibagreenI}[utf8]\color{black}\{inputenc\} \\
\color{nounibaredI}\color{nounibaredI}\textbackslash usepackage\color{black}\color{nounibagreenI}[T1]\color{black}\{fontenc\} \\
\color{nounibaredI}\color{nounibaredI}\textbackslash usepackage\color{black}\{babel\} \\
\color{nounibaredI}\color{unibablueI}\textbackslash\color{unibablueI}begin\color{black}\color{black}\{document\} \\
\color{nounibaredI}\color{nounibaredI}\textbackslash tableofcontents\color{black} \\
\color{nounibaredI}\color{nounibaredI}\textbackslash newpage\color{black} \\
\color{nounibaredI}\color{unibablueI}\textbackslash\color{unibablueI}section\color{black}\color{black}\{Kapitel 1\} \\
Hier kommt der erste Teil. \\
\color{nounibaredI}\color{unibablueI}\textbackslash\color{unibablueI}subsection\color{black}\color{black}\{Unterkapitel 1\} \\
Das erste Unterkapitel. \\
\color{nounibaredI}\color{unibablueI}\textbackslash\color{unibablueI}subsection\color{black}\color{black}\{Unterkapitel 2\} \\
Und noch ein Unterkapitel. \\
\color{nounibaredI}\color{unibablueI}\textbackslash\color{unibablueI}subsubsection\color{black}\color{black}\{Unterunterkapitel 1\} \\
Das ist ein Unterkapitel von einem Unterkapitel. \\
\color{nounibaredI}\color{unibablueI}\textbackslash\color{unibablueI}end\color{black}\color{black}\{document\} \\

\end{ttfamily}
\end{column}
\begin{column}{0.5\textwidth}
\begin{ttfamily}\color{nounibaredI}\textbackslash newpage\color{black}\end{ttfamily}\\
pagebreak\\[3mm]
\begin{ttfamily}\color{unibablueI}\textbackslash section\color{black}\{Title\}\end{ttfamily}\\
A new section begins with the title specified in \begin{ttfamily}\{\}\end{ttfamily}.\\[3mm]
\begin{ttfamily}\color{unibablueI}\textbackslash subsection\color{black}\{Title\}\end{ttfamily}\\
A subsection.\\[3mm]
\begin{ttfamily}\color{unibablueI}\textbackslash subsubsection\color{black}\{Title\}\end{ttfamily}\\
Another level deeper.\\
\end{column}
\end{columns}
\end{frame}

\begin{frame}
\frametitle{Sections}
\framesubtitle{Chapters .PDF}
\begin{columns}
\begin{column}{0.45\textwidth}
\begin{ttfamily}\scriptsize
\color{nounibaredI}\color{nounibaredI}\textbackslash documentclass\color{black}\color{nounibagreenI}[a4paper, pdftex, 12pt, ngerman]\color{black}\{article\} \\
\color{nounibaredI}\color{nounibaredI}\textbackslash usepackage\color{black}\color{nounibagreenI}[utf8]\color{black}\{inputenc\} \\
\color{nounibaredI}\color{nounibaredI}\textbackslash usepackage\color{black}\color{nounibagreenI}[T1]\color{black}\{fontenc\} \\
\color{nounibaredI}\color{nounibaredI}\textbackslash usepackage\color{black}\{babel\} \\
\color{nounibaredI}\color{unibablueI}\textbackslash\color{unibablueI}begin\color{black}\color{black}\{document\} \\
\color{nounibaredI}\color{nounibaredI}\textbackslash tableofcontents\color{black} \\
\color{nounibaredI}\color{nounibaredI}\textbackslash newpage\color{black} \\
\color{nounibaredI}\color{unibablueI}\textbackslash\color{unibablueI}section\color{black}\color{black}\{Kapitel 1\} \\
Hier kommt der erste Teil. \\
\color{nounibaredI}\color{unibablueI}\textbackslash\color{unibablueI}subsection\color{black}\color{black}\{Unterkapitel 1\} \\
Das erste Unterkapitel. \\
\color{nounibaredI}\color{unibablueI}\textbackslash\color{unibablueI}subsection\color{black}\color{black}\{Unterkapitel 2\} \\
Und noch ein Unterkapitel. \\
\color{nounibaredI}\color{unibablueI}\textbackslash\color{unibablueI}subsubsection\color{black}\color{black}\{Unterunterkapitel 1\} \\
Das ist ein Unterkapitel von einem Unterkapitel. \\
\color{nounibaredI}\color{unibablueI}\textbackslash\color{unibablueI}end\color{black}\color{black}\{document\} \\

\end{ttfamily}
\end{column}
\begin{column}{0.55\textwidth}
\image{0.7\textwidth}{image/chapters.png}{The chapters are counted automatically}{img:chapters}
\end{column}
\end{columns}
\end{frame}


\begin{frame}
\frametitle{Sections}
\framesubtitle{Part \& Chapter}
Besides \begin{ttfamily}\color{unibablueI}\textbackslash section\color{black}\{\},
\color{unibablueI}\textbackslash subsection\color{black}\{\}\end{ttfamily}, and \begin{ttfamily}\color{unibablueI}\textbackslash subsubsection\color{black}\{\}\end{ttfamily}, there is also the command
 \begin{ttfamily}\color{unibablueI}\textbackslash part\color{black}\{\}\end{ttfamily} which defines a bigger part.
\begin{ttfamily}\color{unibablueI}\textbackslash part\color{black}\{\}\end{ttfamily} fills a whole page on its own.\\
Besides the document type {\ttfamily article}, there are some others for continous
text documents like {\ttfamily book} and {\ttfamily report}.\\
{\ttfamily book} normally distinguishes between left and right side, 
i.e., whether the page number is left or right, and also the other information that can be contained in header and/or footer.
{\ttfamily book} and {\ttfamily report} have the outline
command \begin{ttfamily}\color{unibablueI}\textbackslash chapter\color{black}\{\}\end{ttfamily}.

%\begin{columns}
%\begin{column}{0.5\textwidth}
%CODE
%\end{column}
%\begin{column}{0.5\textwidth}
%OUTPUT
%\end{column}
%\end{columns}
\end{frame}


\begin{frame}
\frametitle{Formattings}
\framesubtitle{Bold, Italic, Underlined}
\begin{columns}
\begin{column}{0.45\textwidth}
\begin{ttfamily}\scriptsize
\color{nounibaredI}\color{nounibaredI}\textbackslash documentclass\color{black}\color{nounibagreenI}[a4paper, pdftex, 12pt, ngerman]\color{black}\{article\} \\
\color{nounibaredI}\color{nounibaredI}\textbackslash usepackage\color{black}\color{nounibagreenI}[utf8]\color{black}\{inputenc\} \\
\color{nounibaredI}\color{nounibaredI}\textbackslash usepackage\color{black}\color{nounibagreenI}[T1]\color{black}\{fontenc\} \\
\color{nounibaredI}\color{unibablueI}\textbackslash\color{unibablueI}begin\color{black}\color{black}\{document\} \\
Among others there are the following options:\color{nounibaredI}\color{nounibaredI}\textbackslash \color{nounibaredI}\textbackslash \color{black} \\
\color{nounibaredI}\color{nounibaredI}\textbackslash textbf\color{black}\{bold\}\color{nounibaredI}\color{nounibaredI}\textbackslash \color{nounibaredI}\textbackslash \color{black} \\
\color{nounibaredI}\color{nounibaredI}\textbackslash textit\color{black}\{italic\}\color{nounibaredI}\color{nounibaredI}\textbackslash \color{nounibaredI}\textbackslash \color{black} \\
\color{nounibaredI}\color{nounibaredI}\textbackslash underline\color{black}\{underlined\}\color{nounibaredI}\color{nounibaredI}\textbackslash \color{nounibaredI}\textbackslash \color{black} \\
\color{nounibaredI}\color{nounibaredI}\textbackslash underline\color{black}\{\color{nounibaredI}\color{nounibaredI}\textbackslash textbf\color{black}\{underlined and bold\}\}\color{nounibaredI}\color{nounibaredI}\textbackslash \color{nounibaredI}\textbackslash \color{black} \\
\color{nounibaredI}\color{unibablueI}\textbackslash\color{unibablueI}end\color{black}\color{black}\{document\} \\

\end{ttfamily}
\end{column}
\begin{column}{0.55\textwidth}
Among others there are the following options:\\[3mm]
\textbf{bold}\\
\textit{italic}\\
\underline{underlined}\\
\underline{\textbf{underlined and bold}}
%\input{formats_pdf.tex}
\begin{block}{Textformatting}
\begin{ttfamily}$\color{nounibaredII}\backslash$\color{nounibaredII}textbf\color{black}\{text\}\end{ttfamily}
bold text\\
\begin{ttfamily}$\color{nounibaredII}\backslash$\color{nounibaredII}textit\color{black}\{text\}\end{ttfamily}
italic text\\
\begin{ttfamily}$\color{nounibaredII}\backslash$\color{nounibaredII}underline\color{black}\{text\}\end{ttfamily}
underlined
\end{block}
\end{column}
\end{columns}
\end{frame}

\begin{frame}
\frametitle{Formattings}
\framesubtitle{Font Size}
\begin{columns}
\begin{column}{0.5\textwidth}
\begin{ttfamily}\scriptsize
\color{nounibaredI}\color{nounibaredI}\textbackslash documentclass\color{black}\color{nounibagreenI}[a4paper, pdftex, 12pt,ngerman]\color{black}\{article\} \\
\color{nounibaredI}\color{nounibaredI}\textbackslash usepackage\color{black}\color{nounibagreenI}[utf8]\color{black}\{inputenc\} \\
\color{nounibaredI}\color{nounibaredI}\textbackslash usepackage\color{black}\color{nounibagreenI}[T1]\color{black}\{fontenc\} \\
\color{nounibaredI}\color{nounibaredI}\textbackslash usepackage\color{black}\{babel\} \\
\color{nounibaredI}\color{unibablueI}\textbackslash\color{unibablueI}begin\color{black}\color{black}\{document\} \\
\color{nounibaredI}\color{nounibaredI}\textbackslash tiny \color{black} unlesbarer Text \color{nounibaredI}\color{nounibaredI}\textbackslash \color{nounibaredI}\textbackslash \color{black} \\
\color{nounibaredI}\color{nounibaredI}\textbackslash scriptsize \color{black} sehr kleiner Text\color{nounibaredI}\color{nounibaredI}\textbackslash \color{nounibaredI}\textbackslash \color{black} \\
\color{nounibaredI}\color{nounibaredI}\textbackslash footnotesize \color{black} Fussnotengröße \color{nounibaredI}\color{nounibaredI}\textbackslash \color{nounibaredI}\textbackslash \color{black} \\
\color{nounibaredI}\color{nounibaredI}\textbackslash normalsize \color{black} Standartgröße \color{nounibaredI}\color{nounibaredI}\textbackslash \color{nounibaredI}\textbackslash \color{black} \\
\color{nounibaredI}\color{nounibaredI}\textbackslash large \color{black} größer\color{nounibaredI}\color{nounibaredI}\textbackslash \color{nounibaredI}\textbackslash \color{black} \\
\color{nounibaredI}\color{nounibaredI}\textbackslash Large \color{black} noch größer \color{nounibaredI}\color{nounibaredI}\textbackslash \color{nounibaredI}\textbackslash \color{black} \\
\color{nounibaredI}\color{nounibaredI}\textbackslash LARGE \color{black} sehr Groß \color{nounibaredI}\color{nounibaredI}\textbackslash \color{nounibaredI}\textbackslash \color{black} \\
\color{nounibaredI}\color{nounibaredI}\textbackslash huge \color{black} riesig \color{nounibaredI}\color{nounibaredI}\textbackslash \color{nounibaredI}\textbackslash \color{black} \\
\color{nounibaredI}\color{unibablueI}\textbackslash\color{unibablueI}end\color{black}\color{black}\{document\} \\

\end{ttfamily}
\end{column}
\begin{column}{0.5\textwidth}
\rm \tiny ~illegible text\\
\scriptsize ~very small text\\
\footnotesize ~footnote size\\
\normalsize ~standard size\\
\large ~bigger\\
\Large ~even bigger\\
\LARGE ~very big\\
\huge  ~gargantuan\\
\end{column}
\end{columns}
\end{frame}

\begin{frame}
\frametitle{Formattings}
\framesubtitle{Bold, Italic, Underlined}
\begin{columns}
\begin{column}{0.5\textwidth}
\begin{ttfamily}\scriptsize\color{nounibaredII}\textbackslash documentclass\color{nounibagreenI}[a4paper, pdftex, 12pt, ngerman]\color{black}\{article\}\\[3mm] 
$\color{nounibaredII}\backslash$\color{nounibaredII}usepackage\color{nounibagreenI}[utf8]\color{black}\{inputenc\}\\
$\color{nounibaredII}\backslash$\color{nounibaredII}usepackage\color{nounibagreenI}[T1]\color{black}\{fontenc\}\\
$\color{nounibaredII}\backslash$\color{nounibaredII}usepackage\color{nounibagreenI}[iso]\color{black}\{umlaute\}\\
$\color{nounibaredII}\backslash$\color{nounibaredII}usepackage\color{black}\{babel\}\\
\color{gray}\% NEU NEU NEU\\
$\color{nounibaredII}\backslash$\color{nounibaredII}usepackage\color{black}\{eurosym\}\\
$\color{unibablueI}\backslash$\color{unibablueI}begin\color{black}\{document\}\\
$\color{nounibaredII}\backslash$\color{nounibaredII}textit\color{black}\{Some
special characters:\}\\
\color{nounibaredII}\textbackslash \% \textbackslash \$ \textbackslash \& \textbackslash \{ \textbackslash \}
\textbackslash \_ \textbackslash \# \textbackslash S \textbackslash copyright\\
\textbackslash slash \~ ~ \color{unibayellowI}\$\color{nounibaredII}$\color{nounibaredII}\backslash$backslash\color{unibayellowI}\$\color{nounibaredII}  ~\textbackslash euro \\

$\color{nounibaredII}\backslash$\color{nounibaredII}textit\color{black}\{hyphen and dash:\} \\
- -- --- \color{unibayellowI}\$\color{black}-\color{unibayellowI}\$\color{black} (the last one is the mathematical minus) \\

$\color{nounibaredII}\backslash$\color{nounibaredII}textit\color{black}\{quotation marks from \begin{ttfamily}ngerman\end{ttfamily}:\} \\
\color{nounibaredII}\textbackslash glqq \textbackslash grqq \textbackslash flqq \textbackslash frqq\\
\color{unibablueI}\textbackslash end\color{black}\{document\}
\end{ttfamily}
\end{column}
\begin{column}{0.5\textwidth}
\textit{Some special characters:}    \\
\% \$ \& \{ \} \_ \# \S ~ \copyright \slash ~ \textbackslash  \euro \\

\textit{hyphen and dash:} \\
- -- --- $-$ (the last one is the mathematical minus) \\

\textit{quotation marks from (n)german:} \\
\glqq \grqq \flqq \frqq\\[5mm]
For the \euro -sign, you need the package \begin{ttfamily}eurosym\end{ttfamily}.\\

\end{column}
\end{columns}
\medskip
\footnotesize Special characters have to be introduced with '\color{nounibaredI}\textbackslash \color{black}'.
Sometimes, e.g. in headlines, vowel mutations from the package ngerman  have to be built with \grqq a \grqq o
\grqq u and \ss ~with \color{nounibaredI} \textbackslash ss \color{black}, for the rest it is sufficient to include the package {\ttfamily babel}.
\end{frame}
